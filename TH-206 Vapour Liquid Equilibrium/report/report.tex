\DeclareUnicodeCharacter{2212}{-}
\documentclass[11pt]{article}

    \usepackage[breakable]{tcolorbox}
    \usepackage{parskip} % Stop auto-indenting (to mimic markdown behaviour)
    
    \usepackage{iftex}
    \ifPDFTeX
    	\usepackage[T1]{fontenc}
    	\usepackage{mathpazo}
    \else
    	\usepackage{fontspec}
    \fi

    % Basic figure setup, for now with no caption control since it's done
    % automatically by Pandoc (which extracts ![](path) syntax from Markdown).
    \usepackage{graphicx}
    % Maintain compatibility with old templates. Remove in nbconvert 6.0
    \let\Oldincludegraphics\includegraphics
    % Ensure that by default, figures have no caption (until we provide a
    % proper Figure object with a Caption API and a way to capture that
    % in the conversion process - todo).
    \usepackage{caption}
    \DeclareCaptionFormat{nocaption}{}
    \captionsetup{format=nocaption,aboveskip=0pt,belowskip=0pt}

    \usepackage{float}
    \floatplacement{figure}{H} % forces figures to be placed at the correct location
    \usepackage{xcolor} % Allow colors to be defined
    \usepackage{enumerate} % Needed for markdown enumerations to work
    \usepackage{geometry} % Used to adjust the document margins
    \usepackage{amsmath} % Equations
    \usepackage{amssymb} % Equations
    \usepackage{textcomp} % defines textquotesingle
    % Hack from http://tex.stackexchange.com/a/47451/13684:
    \AtBeginDocument{%
        \def\PYZsq{\textquotesingle}% Upright quotes in Pygmentized code
    }
    \usepackage{upquote} % Upright quotes for verbatim code
    \usepackage{eurosym} % defines \euro
    \usepackage[mathletters]{ucs} % Extended unicode (utf-8) support
    \usepackage{fancyvrb} % verbatim replacement that allows latex
    \usepackage{grffile} % extends the file name processing of package graphics 
                         % to support a larger range
    \makeatletter % fix for old versions of grffile with XeLaTeX
    \@ifpackagelater{grffile}{2019/11/01}
    {
      % Do nothing on new versions
    }
    {
      \def\Gread@@xetex#1{%
        \IfFileExists{"\Gin@base".bb}%
        {\Gread@eps{\Gin@base.bb}}%
        {\Gread@@xetex@aux#1}%
      }
    }
    \makeatother
    \usepackage[Export]{adjustbox} % Used to constrain images to a maximum size
    \adjustboxset{max size={0.9\linewidth}{0.9\paperheight}}

    % The hyperref package gives us a pdf with properly built
    % internal navigation ('pdf bookmarks' for the table of contents,
    % internal cross-reference links, web links for URLs, etc.)
    \usepackage{hyperref}
    % The default LaTeX title has an obnoxious amount of whitespace. By default,
    % titling removes some of it. It also provides customization options.
    \usepackage{titling}
    \usepackage{longtable} % longtable support required by pandoc >1.10
    \usepackage{booktabs}  % table support for pandoc > 1.12.2
    \usepackage[inline]{enumitem} % IRkernel/repr support (it uses the enumerate* environment)
    \usepackage[normalem]{ulem} % ulem is needed to support strikethroughs (\sout)
                                % normalem makes italics be italics, not underlines
    \usepackage{mathrsfs}
    

    
    % Colors for the hyperref package
    \definecolor{urlcolor}{rgb}{0,.145,.698}
    \definecolor{linkcolor}{rgb}{.71,0.21,0.01}
    \definecolor{citecolor}{rgb}{.12,.54,.11}

    % ANSI colors
    \definecolor{ansi-black}{HTML}{3E424D}
    \definecolor{ansi-black-intense}{HTML}{282C36}
    \definecolor{ansi-red}{HTML}{E75C58}
    \definecolor{ansi-red-intense}{HTML}{B22B31}
    \definecolor{ansi-green}{HTML}{00A250}
    \definecolor{ansi-green-intense}{HTML}{007427}
    \definecolor{ansi-yellow}{HTML}{DDB62B}
    \definecolor{ansi-yellow-intense}{HTML}{B27D12}
    \definecolor{ansi-blue}{HTML}{208FFB}
    \definecolor{ansi-blue-intense}{HTML}{0065CA}
    \definecolor{ansi-magenta}{HTML}{D160C4}
    \definecolor{ansi-magenta-intense}{HTML}{A03196}
    \definecolor{ansi-cyan}{HTML}{60C6C8}
    \definecolor{ansi-cyan-intense}{HTML}{258F8F}
    \definecolor{ansi-white}{HTML}{C5C1B4}
    \definecolor{ansi-white-intense}{HTML}{A1A6B2}
    \definecolor{ansi-default-inverse-fg}{HTML}{FFFFFF}
    \definecolor{ansi-default-inverse-bg}{HTML}{000000}

    % common color for the border for error outputs.
    \definecolor{outerrorbackground}{HTML}{FFDFDF}

    % commands and environments needed by pandoc snippets
    % extracted from the output of `pandoc -s`
    \providecommand{\tightlist}{%
      \setlength{\itemsep}{0pt}\setlength{\parskip}{0pt}}
    \DefineVerbatimEnvironment{Highlighting}{Verbatim}{commandchars=\\\{\}}
    % Add ',fontsize=\small' for more characters per line
    \newenvironment{Shaded}{}{}
    \newcommand{\KeywordTok}[1]{\textcolor[rgb]{0.00,0.44,0.13}{\textbf{{#1}}}}
    \newcommand{\DataTypeTok}[1]{\textcolor[rgb]{0.56,0.13,0.00}{{#1}}}
    \newcommand{\DecValTok}[1]{\textcolor[rgb]{0.25,0.63,0.44}{{#1}}}
    \newcommand{\BaseNTok}[1]{\textcolor[rgb]{0.25,0.63,0.44}{{#1}}}
    \newcommand{\FloatTok}[1]{\textcolor[rgb]{0.25,0.63,0.44}{{#1}}}
    \newcommand{\CharTok}[1]{\textcolor[rgb]{0.25,0.44,0.63}{{#1}}}
    \newcommand{\StringTok}[1]{\textcolor[rgb]{0.25,0.44,0.63}{{#1}}}
    \newcommand{\CommentTok}[1]{\textcolor[rgb]{0.38,0.63,0.69}{\textit{{#1}}}}
    \newcommand{\OtherTok}[1]{\textcolor[rgb]{0.00,0.44,0.13}{{#1}}}
    \newcommand{\AlertTok}[1]{\textcolor[rgb]{1.00,0.00,0.00}{\textbf{{#1}}}}
    \newcommand{\FunctionTok}[1]{\textcolor[rgb]{0.02,0.16,0.49}{{#1}}}
    \newcommand{\RegionMarkerTok}[1]{{#1}}
    \newcommand{\ErrorTok}[1]{\textcolor[rgb]{1.00,0.00,0.00}{\textbf{{#1}}}}
    \newcommand{\NormalTok}[1]{{#1}}
    
    % Additional commands for more recent versions of Pandoc
    \newcommand{\ConstantTok}[1]{\textcolor[rgb]{0.53,0.00,0.00}{{#1}}}
    \newcommand{\SpecialCharTok}[1]{\textcolor[rgb]{0.25,0.44,0.63}{{#1}}}
    \newcommand{\VerbatimStringTok}[1]{\textcolor[rgb]{0.25,0.44,0.63}{{#1}}}
    \newcommand{\SpecialStringTok}[1]{\textcolor[rgb]{0.73,0.40,0.53}{{#1}}}
    \newcommand{\ImportTok}[1]{{#1}}
    \newcommand{\DocumentationTok}[1]{\textcolor[rgb]{0.73,0.13,0.13}{\textit{{#1}}}}
    \newcommand{\AnnotationTok}[1]{\textcolor[rgb]{0.38,0.63,0.69}{\textbf{\textit{{#1}}}}}
    \newcommand{\CommentVarTok}[1]{\textcolor[rgb]{0.38,0.63,0.69}{\textbf{\textit{{#1}}}}}
    \newcommand{\VariableTok}[1]{\textcolor[rgb]{0.10,0.09,0.49}{{#1}}}
    \newcommand{\ControlFlowTok}[1]{\textcolor[rgb]{0.00,0.44,0.13}{\textbf{{#1}}}}
    \newcommand{\OperatorTok}[1]{\textcolor[rgb]{0.40,0.40,0.40}{{#1}}}
    \newcommand{\BuiltInTok}[1]{{#1}}
    \newcommand{\ExtensionTok}[1]{{#1}}
    \newcommand{\PreprocessorTok}[1]{\textcolor[rgb]{0.74,0.48,0.00}{{#1}}}
    \newcommand{\AttributeTok}[1]{\textcolor[rgb]{0.49,0.56,0.16}{{#1}}}
    \newcommand{\InformationTok}[1]{\textcolor[rgb]{0.38,0.63,0.69}{\textbf{\textit{{#1}}}}}
    \newcommand{\WarningTok}[1]{\textcolor[rgb]{0.38,0.63,0.69}{\textbf{\textit{{#1}}}}}
    
    
    % Define a nice break command that doesn't care if a line doesn't already
    % exist.
    \def\br{\hspace*{\fill} \\* }
    % Math Jax compatibility definitions
    \def\gt{>}
    \def\lt{<}
    \let\Oldtex\TeX
    \let\Oldlatex\LaTeX
    \renewcommand{\TeX}{\textrm{\Oldtex}}
    \renewcommand{\LaTeX}{\textrm{\Oldlatex}}
    % Document parameters
    % Document title
    \title{report}
    
    
    
    
    
% Pygments definitions
\makeatletter
\def\PY@reset{\let\PY@it=\relax \let\PY@bf=\relax%
    \let\PY@ul=\relax \let\PY@tc=\relax%
    \let\PY@bc=\relax \let\PY@ff=\relax}
\def\PY@tok#1{\csname PY@tok@#1\endcsname}
\def\PY@toks#1+{\ifx\relax#1\empty\else%
    \PY@tok{#1}\expandafter\PY@toks\fi}
\def\PY@do#1{\PY@bc{\PY@tc{\PY@ul{%
    \PY@it{\PY@bf{\PY@ff{#1}}}}}}}
\def\PY#1#2{\PY@reset\PY@toks#1+\relax+\PY@do{#2}}

\@namedef{PY@tok@w}{\def\PY@tc##1{\textcolor[rgb]{0.73,0.73,0.73}{##1}}}
\@namedef{PY@tok@c}{\let\PY@it=\textit\def\PY@tc##1{\textcolor[rgb]{0.25,0.50,0.50}{##1}}}
\@namedef{PY@tok@cp}{\def\PY@tc##1{\textcolor[rgb]{0.74,0.48,0.00}{##1}}}
\@namedef{PY@tok@k}{\let\PY@bf=\textbf\def\PY@tc##1{\textcolor[rgb]{0.00,0.50,0.00}{##1}}}
\@namedef{PY@tok@kp}{\def\PY@tc##1{\textcolor[rgb]{0.00,0.50,0.00}{##1}}}
\@namedef{PY@tok@kt}{\def\PY@tc##1{\textcolor[rgb]{0.69,0.00,0.25}{##1}}}
\@namedef{PY@tok@o}{\def\PY@tc##1{\textcolor[rgb]{0.40,0.40,0.40}{##1}}}
\@namedef{PY@tok@ow}{\let\PY@bf=\textbf\def\PY@tc##1{\textcolor[rgb]{0.67,0.13,1.00}{##1}}}
\@namedef{PY@tok@nb}{\def\PY@tc##1{\textcolor[rgb]{0.00,0.50,0.00}{##1}}}
\@namedef{PY@tok@nf}{\def\PY@tc##1{\textcolor[rgb]{0.00,0.00,1.00}{##1}}}
\@namedef{PY@tok@nc}{\let\PY@bf=\textbf\def\PY@tc##1{\textcolor[rgb]{0.00,0.00,1.00}{##1}}}
\@namedef{PY@tok@nn}{\let\PY@bf=\textbf\def\PY@tc##1{\textcolor[rgb]{0.00,0.00,1.00}{##1}}}
\@namedef{PY@tok@ne}{\let\PY@bf=\textbf\def\PY@tc##1{\textcolor[rgb]{0.82,0.25,0.23}{##1}}}
\@namedef{PY@tok@nv}{\def\PY@tc##1{\textcolor[rgb]{0.10,0.09,0.49}{##1}}}
\@namedef{PY@tok@no}{\def\PY@tc##1{\textcolor[rgb]{0.53,0.00,0.00}{##1}}}
\@namedef{PY@tok@nl}{\def\PY@tc##1{\textcolor[rgb]{0.63,0.63,0.00}{##1}}}
\@namedef{PY@tok@ni}{\let\PY@bf=\textbf\def\PY@tc##1{\textcolor[rgb]{0.60,0.60,0.60}{##1}}}
\@namedef{PY@tok@na}{\def\PY@tc##1{\textcolor[rgb]{0.49,0.56,0.16}{##1}}}
\@namedef{PY@tok@nt}{\let\PY@bf=\textbf\def\PY@tc##1{\textcolor[rgb]{0.00,0.50,0.00}{##1}}}
\@namedef{PY@tok@nd}{\def\PY@tc##1{\textcolor[rgb]{0.67,0.13,1.00}{##1}}}
\@namedef{PY@tok@s}{\def\PY@tc##1{\textcolor[rgb]{0.73,0.13,0.13}{##1}}}
\@namedef{PY@tok@sd}{\let\PY@it=\textit\def\PY@tc##1{\textcolor[rgb]{0.73,0.13,0.13}{##1}}}
\@namedef{PY@tok@si}{\let\PY@bf=\textbf\def\PY@tc##1{\textcolor[rgb]{0.73,0.40,0.53}{##1}}}
\@namedef{PY@tok@se}{\let\PY@bf=\textbf\def\PY@tc##1{\textcolor[rgb]{0.73,0.40,0.13}{##1}}}
\@namedef{PY@tok@sr}{\def\PY@tc##1{\textcolor[rgb]{0.73,0.40,0.53}{##1}}}
\@namedef{PY@tok@ss}{\def\PY@tc##1{\textcolor[rgb]{0.10,0.09,0.49}{##1}}}
\@namedef{PY@tok@sx}{\def\PY@tc##1{\textcolor[rgb]{0.00,0.50,0.00}{##1}}}
\@namedef{PY@tok@m}{\def\PY@tc##1{\textcolor[rgb]{0.40,0.40,0.40}{##1}}}
\@namedef{PY@tok@gh}{\let\PY@bf=\textbf\def\PY@tc##1{\textcolor[rgb]{0.00,0.00,0.50}{##1}}}
\@namedef{PY@tok@gu}{\let\PY@bf=\textbf\def\PY@tc##1{\textcolor[rgb]{0.50,0.00,0.50}{##1}}}
\@namedef{PY@tok@gd}{\def\PY@tc##1{\textcolor[rgb]{0.63,0.00,0.00}{##1}}}
\@namedef{PY@tok@gi}{\def\PY@tc##1{\textcolor[rgb]{0.00,0.63,0.00}{##1}}}
\@namedef{PY@tok@gr}{\def\PY@tc##1{\textcolor[rgb]{1.00,0.00,0.00}{##1}}}
\@namedef{PY@tok@ge}{\let\PY@it=\textit}
\@namedef{PY@tok@gs}{\let\PY@bf=\textbf}
\@namedef{PY@tok@gp}{\let\PY@bf=\textbf\def\PY@tc##1{\textcolor[rgb]{0.00,0.00,0.50}{##1}}}
\@namedef{PY@tok@go}{\def\PY@tc##1{\textcolor[rgb]{0.53,0.53,0.53}{##1}}}
\@namedef{PY@tok@gt}{\def\PY@tc##1{\textcolor[rgb]{0.00,0.27,0.87}{##1}}}
\@namedef{PY@tok@err}{\def\PY@bc##1{{\setlength{\fboxsep}{\string -\fboxrule}\fcolorbox[rgb]{1.00,0.00,0.00}{1,1,1}{\strut ##1}}}}
\@namedef{PY@tok@kc}{\let\PY@bf=\textbf\def\PY@tc##1{\textcolor[rgb]{0.00,0.50,0.00}{##1}}}
\@namedef{PY@tok@kd}{\let\PY@bf=\textbf\def\PY@tc##1{\textcolor[rgb]{0.00,0.50,0.00}{##1}}}
\@namedef{PY@tok@kn}{\let\PY@bf=\textbf\def\PY@tc##1{\textcolor[rgb]{0.00,0.50,0.00}{##1}}}
\@namedef{PY@tok@kr}{\let\PY@bf=\textbf\def\PY@tc##1{\textcolor[rgb]{0.00,0.50,0.00}{##1}}}
\@namedef{PY@tok@bp}{\def\PY@tc##1{\textcolor[rgb]{0.00,0.50,0.00}{##1}}}
\@namedef{PY@tok@fm}{\def\PY@tc##1{\textcolor[rgb]{0.00,0.00,1.00}{##1}}}
\@namedef{PY@tok@vc}{\def\PY@tc##1{\textcolor[rgb]{0.10,0.09,0.49}{##1}}}
\@namedef{PY@tok@vg}{\def\PY@tc##1{\textcolor[rgb]{0.10,0.09,0.49}{##1}}}
\@namedef{PY@tok@vi}{\def\PY@tc##1{\textcolor[rgb]{0.10,0.09,0.49}{##1}}}
\@namedef{PY@tok@vm}{\def\PY@tc##1{\textcolor[rgb]{0.10,0.09,0.49}{##1}}}
\@namedef{PY@tok@sa}{\def\PY@tc##1{\textcolor[rgb]{0.73,0.13,0.13}{##1}}}
\@namedef{PY@tok@sb}{\def\PY@tc##1{\textcolor[rgb]{0.73,0.13,0.13}{##1}}}
\@namedef{PY@tok@sc}{\def\PY@tc##1{\textcolor[rgb]{0.73,0.13,0.13}{##1}}}
\@namedef{PY@tok@dl}{\def\PY@tc##1{\textcolor[rgb]{0.73,0.13,0.13}{##1}}}
\@namedef{PY@tok@s2}{\def\PY@tc##1{\textcolor[rgb]{0.73,0.13,0.13}{##1}}}
\@namedef{PY@tok@sh}{\def\PY@tc##1{\textcolor[rgb]{0.73,0.13,0.13}{##1}}}
\@namedef{PY@tok@s1}{\def\PY@tc##1{\textcolor[rgb]{0.73,0.13,0.13}{##1}}}
\@namedef{PY@tok@mb}{\def\PY@tc##1{\textcolor[rgb]{0.40,0.40,0.40}{##1}}}
\@namedef{PY@tok@mf}{\def\PY@tc##1{\textcolor[rgb]{0.40,0.40,0.40}{##1}}}
\@namedef{PY@tok@mh}{\def\PY@tc##1{\textcolor[rgb]{0.40,0.40,0.40}{##1}}}
\@namedef{PY@tok@mi}{\def\PY@tc##1{\textcolor[rgb]{0.40,0.40,0.40}{##1}}}
\@namedef{PY@tok@il}{\def\PY@tc##1{\textcolor[rgb]{0.40,0.40,0.40}{##1}}}
\@namedef{PY@tok@mo}{\def\PY@tc##1{\textcolor[rgb]{0.40,0.40,0.40}{##1}}}
\@namedef{PY@tok@ch}{\let\PY@it=\textit\def\PY@tc##1{\textcolor[rgb]{0.25,0.50,0.50}{##1}}}
\@namedef{PY@tok@cm}{\let\PY@it=\textit\def\PY@tc##1{\textcolor[rgb]{0.25,0.50,0.50}{##1}}}
\@namedef{PY@tok@cpf}{\let\PY@it=\textit\def\PY@tc##1{\textcolor[rgb]{0.25,0.50,0.50}{##1}}}
\@namedef{PY@tok@c1}{\let\PY@it=\textit\def\PY@tc##1{\textcolor[rgb]{0.25,0.50,0.50}{##1}}}
\@namedef{PY@tok@cs}{\let\PY@it=\textit\def\PY@tc##1{\textcolor[rgb]{0.25,0.50,0.50}{##1}}}

\def\PYZbs{\char`\\}
\def\PYZus{\char`\_}
\def\PYZob{\char`\{}
\def\PYZcb{\char`\}}
\def\PYZca{\char`\^}
\def\PYZam{\char`\&}
\def\PYZlt{\char`\<}
\def\PYZgt{\char`\>}
\def\PYZsh{\char`\#}
\def\PYZpc{\char`\%}
\def\PYZdl{\char`\$}
\def\PYZhy{\char`\-}
\def\PYZsq{\char`\'}
\def\PYZdq{\char`\"}
\def\PYZti{\char`\~}
% for compatibility with earlier versions
\def\PYZat{@}
\def\PYZlb{[}
\def\PYZrb{]}
\makeatother


    % For linebreaks inside Verbatim environment from package fancyvrb. 
    \makeatletter
        \newbox\Wrappedcontinuationbox 
        \newbox\Wrappedvisiblespacebox 
        \newcommand*\Wrappedvisiblespace {\textcolor{red}{\textvisiblespace}} 
        \newcommand*\Wrappedcontinuationsymbol {\textcolor{red}{\llap{\tiny$\m@th\hookrightarrow$}}} 
        \newcommand*\Wrappedcontinuationindent {3ex } 
        \newcommand*\Wrappedafterbreak {\kern\Wrappedcontinuationindent\copy\Wrappedcontinuationbox} 
        % Take advantage of the already applied Pygments mark-up to insert 
        % potential linebreaks for TeX processing. 
        %        {, <, #, %, $, ' and ": go to next line. 
        %        _, }, ^, &, >, - and ~: stay at end of broken line. 
        % Use of \textquotesingle for straight quote. 
        \newcommand*\Wrappedbreaksatspecials {% 
            \def\PYGZus{\discretionary{\char`\_}{\Wrappedafterbreak}{\char`\_}}% 
            \def\PYGZob{\discretionary{}{\Wrappedafterbreak\char`\{}{\char`\{}}% 
            \def\PYGZcb{\discretionary{\char`\}}{\Wrappedafterbreak}{\char`\}}}% 
            \def\PYGZca{\discretionary{\char`\^}{\Wrappedafterbreak}{\char`\^}}% 
            \def\PYGZam{\discretionary{\char`\&}{\Wrappedafterbreak}{\char`\&}}% 
            \def\PYGZlt{\discretionary{}{\Wrappedafterbreak\char`\<}{\char`\<}}% 
            \def\PYGZgt{\discretionary{\char`\>}{\Wrappedafterbreak}{\char`\>}}% 
            \def\PYGZsh{\discretionary{}{\Wrappedafterbreak\char`\#}{\char`\#}}% 
            \def\PYGZpc{\discretionary{}{\Wrappedafterbreak\char`\%}{\char`\%}}% 
            \def\PYGZdl{\discretionary{}{\Wrappedafterbreak\char`\$}{\char`\$}}% 
            \def\PYGZhy{\discretionary{\char`\-}{\Wrappedafterbreak}{\char`\-}}% 
            \def\PYGZsq{\discretionary{}{\Wrappedafterbreak\textquotesingle}{\textquotesingle}}% 
            \def\PYGZdq{\discretionary{}{\Wrappedafterbreak\char`\"}{\char`\"}}% 
            \def\PYGZti{\discretionary{\char`\~}{\Wrappedafterbreak}{\char`\~}}% 
        } 
        % Some characters . , ; ? ! / are not pygmentized. 
        % This macro makes them "active" and they will insert potential linebreaks 
        \newcommand*\Wrappedbreaksatpunct {% 
            \lccode`\~`\.\lowercase{\def~}{\discretionary{\hbox{\char`\.}}{\Wrappedafterbreak}{\hbox{\char`\.}}}% 
            \lccode`\~`\,\lowercase{\def~}{\discretionary{\hbox{\char`\,}}{\Wrappedafterbreak}{\hbox{\char`\,}}}% 
            \lccode`\~`\;\lowercase{\def~}{\discretionary{\hbox{\char`\;}}{\Wrappedafterbreak}{\hbox{\char`\;}}}% 
            \lccode`\~`\:\lowercase{\def~}{\discretionary{\hbox{\char`\:}}{\Wrappedafterbreak}{\hbox{\char`\:}}}% 
            \lccode`\~`\?\lowercase{\def~}{\discretionary{\hbox{\char`\?}}{\Wrappedafterbreak}{\hbox{\char`\?}}}% 
            \lccode`\~`\!\lowercase{\def~}{\discretionary{\hbox{\char`\!}}{\Wrappedafterbreak}{\hbox{\char`\!}}}% 
            \lccode`\~`\/\lowercase{\def~}{\discretionary{\hbox{\char`\/}}{\Wrappedafterbreak}{\hbox{\char`\/}}}% 
            \catcode`\.\active
            \catcode`\,\active 
            \catcode`\;\active
            \catcode`\:\active
            \catcode`\?\active
            \catcode`\!\active
            \catcode`\/\active 
            \lccode`\~`\~ 	
        }
    \makeatother

    \let\OriginalVerbatim=\Verbatim
    \makeatletter
    \renewcommand{\Verbatim}[1][1]{%
        %\parskip\z@skip
        \sbox\Wrappedcontinuationbox {\Wrappedcontinuationsymbol}%
        \sbox\Wrappedvisiblespacebox {\FV@SetupFont\Wrappedvisiblespace}%
        \def\FancyVerbFormatLine ##1{\hsize\linewidth
            \vtop{\raggedright\hyphenpenalty\z@\exhyphenpenalty\z@
                \doublehyphendemerits\z@\finalhyphendemerits\z@
                \strut ##1\strut}%
        }%
        % If the linebreak is at a space, the latter will be displayed as visible
        % space at end of first line, and a continuation symbol starts next line.
        % Stretch/shrink are however usually zero for typewriter font.
        \def\FV@Space {%
            \nobreak\hskip\z@ plus\fontdimen3\font minus\fontdimen4\font
            \discretionary{\copy\Wrappedvisiblespacebox}{\Wrappedafterbreak}
            {\kern\fontdimen2\font}%
        }%
        
        % Allow breaks at special characters using \PYG... macros.
        \Wrappedbreaksatspecials
        % Breaks at punctuation characters . , ; ? ! and / need catcode=\active 	
        \OriginalVerbatim[#1,codes*=\Wrappedbreaksatpunct]%
    }
    \makeatother

    % Exact colors from NB
    \definecolor{incolor}{HTML}{303F9F}
    \definecolor{outcolor}{HTML}{D84315}
    \definecolor{cellborder}{HTML}{CFCFCF}
    \definecolor{cellbackground}{HTML}{F7F7F7}
    
    % prompt
    \makeatletter
    \newcommand{\boxspacing}{\kern\kvtcb@left@rule\kern\kvtcb@boxsep}
    \makeatother
    \newcommand{\prompt}[4]{
        {\ttfamily\llap{{\color{#2}[#3]:\hspace{3pt}#4}}\vspace{-\baselineskip}}
    }
    

    
    % Prevent overflowing lines due to hard-to-break entities
    \sloppy 
    % Setup hyperref package
    \hypersetup{
      breaklinks=true,  % so long urls are correctly broken across lines
      colorlinks=true,
      urlcolor=urlcolor,
      linkcolor=linkcolor,
      citecolor=citecolor,
      }
    % Slightly bigger margins than the latex defaults
    
    \geometry{verbose,tmargin=1in,bmargin=1in,lmargin=1in,rmargin=1in}
    
\date{}

\begin{document}

\hypertarget{objectives}{%
  \section{Objectives}\label{objectives}}

\begin{enumerate}
  \def\labelenumi{\arabic{enumi}.}
  \tightlist
  \item
        Determine the vapour-liquid equilibrium data for the acetone (A)
        benzene (B) homogeneous mixture
  \item
        Check the thermodynamic consistency of the data using the Gibbs Duhem
        Equation
  \item
        Present the thermodynamic characteristics and constants of Van-Laar
        equation for the system
\end{enumerate}

\hypertarget{procedure}{%
  \section{Procedure}\label{procedure}}

\begin{enumerate}
  \def\labelenumi{\arabic{enumi}.}
  \tightlist
  \item
        Prepare 9 solutions of acetone and benzene having different volume
        fraction of acetone.
  \item
        Measure the refractive index of the solutions using a refractometer,
        take three readings for each solution of acetone and benzene.
  \item
        Now prepare a solution by adding approximately 480 ml of acetone and
        120 ml of benzene so that the solution level is same as the level of
        cotrell pump.
  \item
        Now pour the solution in the still and put the voltage value to 120V
        and wait for it to reach a steady temperature.
  \item
        Put the voltage value to 80-85V when it starts boiling.
  \item
        Dispose off the liquid and vapor solution collected by cotrell pump
        and condenser i.e purge the obtained liquid and vapor solution once..
  \item
        Collect the liquid and vapour solution and put the voltage value to
        zero and measure their refractive index.
  \item
        Remove some amount of solution from the still and add some benzene to
        it and set the voltage value to 120 and then repeat step 5, 6.
  \item
        Take four such readings by removing some amount of the solution and
        adding the same amount of the benzene to it every time.
\end{enumerate}

\hypertarget{set-up-assembly}{%
  \section{Set-Up Assembly}\label{set-up-assembly}}

\begin{center}
  \adjustimage{max size={0.3\linewidth}{0.3\paperheight}}{images/apparatus.png}
\end{center}
\pagebreak

\hypertarget{results}{%
  \section{Results}\label{results}}

\hypertarget{raw-data}{%
  \subsection{Raw Data}\label{raw-data}}

\textbf{Table 1: Properties of Components}

\begin{center}
  \adjustimage{max size={0.7\linewidth}{0.7\paperheight}}{images/table1.png}
\end{center}

\textbf{Table 2: Calibration Curve Data}

\begin{center}
  \adjustimage{max size={0.9\linewidth}{0.9\paperheight}}{images/table2.png}
\end{center}

\textbf{Table 3: Experimental Data}

\begin{center}
  \adjustimage{max size={0.9\linewidth}{0.9\paperheight}}{images/table3.png}
\end{center}

\pagebreak
\hypertarget{derived-data}{%
  \subsection{Derived Data}\label{derived-data}}

\hypertarget{mol-fraction-from-vol-fraction}{%
  \subsubsection{Mol Fraction from Vol
    Fraction}\label{mol-fraction-from-vol-fraction}}

We calculated the mole fractions from the volume ratios as follows:

\begin{equation}
  \begin{split}
    \frac{V_A}{V_B}&=\beta \:\:\:(\rm say)\\
    \frac{m_A/\rho_A}{m_B/\rho_B}&=\beta\\
    \frac{m_A}{m_B}&=\beta\frac{\rho_A}{\rho_B}\\
    \frac{n_AM_A}{n_BM_B}&=\beta\frac{\rho_A}{\rho_B}\\
    \frac{n_A}{n_B}&=\beta\frac{\rho_A}{\rho_B}\frac{M_B}{M_A}\\
    \implies n_B&=\frac{\rho_BM_A}{\beta\rho_AM_B}n_A\\
    \because x_A&=\frac{n_A}{n_A+n_B},\:\:\rm we\: get:\\
    x_A = \frac{n_A}{n_A+\frac{\rho_BM_A}{\beta\rho_AM_B}n_A}&=\frac{\beta\rho_AM_B}{\beta\rho_AM_B+\rho_BM_A}
  \end{split}
\end{equation}

For the first value of \(\beta=9\), the above formula yielded
\(x_A=0.909\). The \emph{average} refractive index (RI) was plotted
against \(x_A\) and a linear regression was carried out.

\textbf{Table 4: Mole fraction calculated from linear regression}

\begin{center}
  \adjustimage{max size={0.9\linewidth}{0.9\paperheight}}{images/table4.png}
\end{center}

\hypertarget{saturation-pressure-from-antoines-equation}{%
  \subsubsection{Saturation Pressure from Antoine's
    Equation:}\label{saturation-pressure-from-antoines-equation}}

The form of Antoine's equation obtained from Smith and Van-Ness is as
follows:

\begin{equation}
  \ln{P^{sat}}(kPa)=A-\frac{B}{T(^{\circ}C)+C}
\end{equation}

The constants for A (acetone) are: A = 14.3145; B = 2756.22; C =228.060
while for B (benzene) they are: A = 13.7819; B = 2726.81; C = 217.572

\hypertarget{activity-coefficient-from-modified-raoults-law}{%
  \subsubsection{Activity Coefficient from Modified Raoult's
    Law:}\label{activity-coefficient-from-modified-raoults-law}}

The statement of the Modified Raoult's Law reads:

\begin{equation}
  \begin{split}
    x_A\gamma_AP_A^{sat}&=y_A\pi_A\\
    \implies \gamma_A&=\frac{y_A\pi_A}{x_AP_A^{sat}}
  \end{split}
\end{equation}

\textbf{Table 5: Activity Coefficient and Saturation Pressure}

\begin{center}
  \adjustimage{max size={0.9\linewidth}{0.9\paperheight}}{images/table5.png}
\end{center}

\hypertarget{van-laar-constants-from-data}{%
  \subsubsection{Van-Laar Constants from
    Data:}\label{van-laar-constants-from-data}}

The Van-Laar constants are given by:

\begin{equation}
  \begin{split}
    a&=\log{\gamma_A}\left[1+\frac{x_B\log{\gamma_B}}{x_A\log{\gamma_A}}\right]^2\\
    b&=\log{\gamma_B}\left[1+\frac{x_A\log{\gamma_A}}{x_B\log{\gamma_B}}\right]^2
  \end{split}
\end{equation}

\textbf{Table 6: Calculation of excess Gibbs free energy}

\begin{center}
  \adjustimage{max size={0.9\linewidth}{0.9\paperheight}}{images/table6.png}
\end{center}
\pagebreak

\hypertarget{observations}{%
  \section{Observations:}\label{observations}}

\begin{itemize}
  \tightlist
  \item
        The refractive index of the mixture shows a linear relation with the
        volume (and thus mole) fraction of acetone. As the mol fraction of
        acetone decreases, the refractive index increases.
  \item
        As the mole fraction of acetone (more volatile component) decreases,
        the VLE temperature increases.
\end{itemize}

\hypertarget{hypothesis-conclusion}{%
  \section{Hypothesis \& Conclusion:}\label{hypothesis-conclusion}}

One hypothesis made at the start was that data follows thermodynamic
consistency. The way to check that is by the integral test (Gibb's-Duhem
Equation):

\begin{equation}
  \int_{0}^{1}\ln{\frac{\gamma_A}{\gamma_B}}dx_a=0
\end{equation}

To that extent, \(\ln{(\gamma_a/\gamma_b)}\) v/s \(x_a\) was plotted and
the area bound was measured. The value deviated from 0 and thus,
thermodynmic consistency is \textbf{not} followed.

\emph{Some part might be because of experimental error, and it could be
  rectified by taking more data points.}

Another aim was to calculate the Van-Laar constants. For experiment run
(ie each VLE established), the constants were calculated using the
formulae mentioned above. The values came out as:

\textbf{Table 7: Van-Laar Constants calculated from data}

\begin{center}
  \adjustimage{max size={0.6\linewidth}{0.6\paperheight}}{images/table7.png}
\end{center}

It is expected that the Van-Laar constants should stay the same
irrespective of the temperature/mol fraction of the VLE for the same
mixture components. However, there is a clear deviation in both
\(a\:\&\:b\) once again proving thermodynamic inconsistency.

\hypertarget{solve}{%
  \section{Solve}\label{solve}}

\begin{tcolorbox}[breakable, size=fbox, boxrule=1pt, pad at break*=1mm,colback=cellbackground, colframe=cellborder]
  \prompt{In}{incolor}{1}{\boxspacing}
  \begin{Verbatim}[commandchars=\\\{\}]
    \PY{k+kn}{import} \PY{n+nn}{numpy} \PY{k}{as} \PY{n+nn}{np}
    \PY{k+kn}{import} \PY{n+nn}{matplotlib}\PY{n+nn}{.}\PY{n+nn}{pyplot} \PY{k}{as} \PY{n+nn}{plt}
    \PY{k+kn}{from} \PY{n+nn}{sklearn}\PY{n+nn}{.}\PY{n+nn}{linear\PYZus{}model} \PY{k+kn}{import} \PY{n}{LinearRegression}
    \PY{k+kn}{from} \PY{n+nn}{scipy} \PY{k+kn}{import} \PY{n}{interpolate}
  \end{Verbatim}
\end{tcolorbox}

\begin{itemize}
  \tightlist
  \item
        A = Acetone
  \item
        B = Benzene
\end{itemize}

\begin{tcolorbox}[breakable, size=fbox, boxrule=1pt, pad at break*=1mm,colback=cellbackground, colframe=cellborder]
  \prompt{In}{incolor}{2}{\boxspacing}
  \begin{Verbatim}[commandchars=\\\{\}]
    \PY{n}{rho\PYZus{}A} \PY{o}{=} \PY{l+m+mi}{729} \PY{c+c1}{\PYZsh{} kg/m\PYZca{}3}
    \PY{n}{rho\PYZus{}B} \PY{o}{=} \PY{l+m+mi}{879} \PY{c+c1}{\PYZsh{} kg/m\PYZca{}3}

    \PY{c+c1}{\PYZsh{} ? Molecular weight}
    \PY{n}{M\PYZus{}A} \PY{o}{=} \PY{l+m+mf}{58.08} \PY{c+c1}{\PYZsh{} g/mol}
    \PY{n}{M\PYZus{}B} \PY{o}{=} \PY{l+m+mf}{78.11} \PY{c+c1}{\PYZsh{} g/mol}
  \end{Verbatim}
\end{tcolorbox}

\hypertarget{calibration-curve}{%
  \subsection{Calibration curve}\label{calibration-curve}}

\begin{tcolorbox}[breakable, size=fbox, boxrule=1pt, pad at break*=1mm,colback=cellbackground, colframe=cellborder]
  \prompt{In}{incolor}{3}{\boxspacing}
  \begin{Verbatim}[commandchars=\\\{\}]
    \PY{c+c1}{\PYZsh{} Volume ratio}
    \PY{n}{VR} \PY{o}{=} \PY{p}{[}\PY{l+m+mi}{9}\PY{o}{/}\PY{l+m+mi}{1}\PY{p}{,} \PY{l+m+mi}{8}\PY{o}{/}\PY{l+m+mi}{2}\PY{p}{,} \PY{l+m+mi}{7}\PY{o}{/}\PY{l+m+mi}{3}\PY{p}{,} \PY{l+m+mi}{6}\PY{o}{/}\PY{l+m+mi}{4}\PY{p}{,} \PY{l+m+mi}{5}\PY{o}{/}\PY{l+m+mi}{5}\PY{p}{,} \PY{l+m+mi}{4}\PY{o}{/}\PY{l+m+mi}{6}\PY{p}{,} \PY{l+m+mi}{3}\PY{o}{/}\PY{l+m+mi}{7}\PY{p}{,} \PY{l+m+mi}{2}\PY{o}{/}\PY{l+m+mi}{8}\PY{p}{,} \PY{l+m+mi}{1}\PY{o}{/}\PY{l+m+mi}{9}\PY{p}{]}
    \PY{n}{VR} \PY{o}{=} \PY{n}{np}\PY{o}{.}\PY{n}{array}\PY{p}{(}\PY{n}{VR}\PY{p}{)}
    \PY{n}{VR} \PY{o}{=} \PY{n}{VR} \PY{o}{*} \PY{p}{(}\PY{n}{rho\PYZus{}A} \PY{o}{/} \PY{n}{M\PYZus{}A}\PY{p}{)} \PY{o}{/} \PY{p}{(}\PY{n}{rho\PYZus{}B} \PY{o}{/} \PY{n}{M\PYZus{}B}\PY{p}{)}
    \PY{n}{VR\PYZus{}A} \PY{o}{=} \PY{p}{[}\PY{p}{(}\PY{n}{r} \PY{o}{/} \PY{p}{(}\PY{n}{r}\PY{o}{+}\PY{l+m+mf}{1.0}\PY{p}{)}\PY{p}{)} \PY{k}{for} \PY{n}{r} \PY{o+ow}{in} \PY{n}{VR}\PY{p}{]}
    \PY{n}{VR\PYZus{}A} \PY{o}{=} \PY{p}{[}\PY{l+m+mf}{1.0}\PY{p}{,} \PY{o}{*}\PY{n}{VR\PYZus{}A}\PY{p}{,} \PY{l+m+mf}{0.0}\PY{p}{]}

    \PY{c+c1}{\PYZsh{} x = [0, 0.1, 0.2, 0.3, 0.4, 0.5, 0.6, 0.7, 0.8, 0.9, 1]}
    \PY{n}{x\PYZus{}A} \PY{o}{=} \PY{n}{np}\PY{o}{.}\PY{n}{array}\PY{p}{(}\PY{n}{VR\PYZus{}A}\PY{p}{)}
    \PY{n+nb}{print}\PY{p}{(}\PY{l+s+s2}{\PYZdq{}}\PY{l+s+s2}{Mole fraction of Acetone: }\PY{l+s+s2}{\PYZdq{}}\PY{p}{,} \PY{n}{x\PYZus{}A}\PY{p}{)}

    \PY{n}{RI\PYZus{}1} \PY{o}{=} \PY{p}{[}\PY{l+m+mf}{1.351791}\PY{p}{,} \PY{l+m+mf}{1.364765}\PY{p}{,} \PY{l+m+mf}{1.38243}\PY{p}{,} \PY{l+m+mf}{1.400394}\PY{p}{,} \PY{l+m+mf}{1.416661}\PY{p}{,} \PY{l+m+mf}{1.437419}\PY{p}{,}
    \PY{l+m+mf}{1.443108}\PY{p}{,} \PY{l+m+mf}{1.454685}\PY{p}{,} \PY{l+m+mf}{1.471052}\PY{p}{,} \PY{l+m+mf}{1.484525}\PY{p}{,} \PY{l+m+mf}{1.491711}\PY{p}{]}

    \PY{n}{RI\PYZus{}2} \PY{o}{=} \PY{p}{[}\PY{l+m+mf}{1.351891}\PY{p}{,} \PY{l+m+mf}{1.364765}\PY{p}{,} \PY{l+m+mf}{1.382929}\PY{p}{,} \PY{l+m+mf}{1.399296}\PY{p}{,} \PY{l+m+mf}{1.415763}\PY{p}{,}
    \PY{l+m+mf}{1.43732}\PY{p}{,} \PY{l+m+mf}{1.445004}\PY{p}{,} \PY{l+m+mf}{1.4472}\PY{p}{,} \PY{l+m+mf}{1.471551}\PY{p}{,} \PY{l+m+mf}{1.483427}\PY{p}{,} \PY{l+m+mf}{1.492709}\PY{p}{]}

    \PY{n}{RI\PYZus{}3} \PY{o}{=} \PY{p}{[}\PY{l+m+mf}{1.351691}\PY{p}{,} \PY{l+m+mf}{1.364066}\PY{p}{,} \PY{l+m+mf}{1.383827}\PY{p}{,} \PY{l+m+mf}{1.400593}\PY{p}{,} \PY{l+m+mf}{1.417459}\PY{p}{,} \PY{l+m+mf}{1.440513}\PY{p}{,}
    \PY{l+m+mf}{1.443607}\PY{p}{,} \PY{l+m+mf}{1.454884}\PY{p}{,} \PY{l+m+mf}{1.471351}\PY{p}{,} \PY{l+m+mf}{1.484425}\PY{p}{,} \PY{l+m+mf}{1.49201}\PY{p}{]}

    \PY{n}{y} \PY{o}{=} \PY{n}{np}\PY{o}{.}\PY{n}{mean}\PY{p}{(}\PY{p}{[}\PY{n}{RI\PYZus{}1}\PY{p}{,} \PY{n}{RI\PYZus{}2}\PY{p}{,} \PY{n}{RI\PYZus{}3}\PY{p}{]}\PY{p}{,} \PY{n}{axis}\PY{o}{=}\PY{l+m+mi}{0}\PY{p}{)}
  \end{Verbatim}
\end{tcolorbox}

\begin{Verbatim}[commandchars=\\\{\}]
  Mole fraction of Acetone:  [1.         0.90940653 0.81689935 0.72241717
  0.62589606 0.5272693
  0.42646723 0.32341707 0.21804277 0.11026481 0.        ]
\end{Verbatim}

\begin{tcolorbox}[breakable, size=fbox, boxrule=1pt, pad at break*=1mm,colback=cellbackground, colframe=cellborder]
  \prompt{In}{incolor}{4}{\boxspacing}
  \begin{Verbatim}[commandchars=\\\{\}]
    \PY{n}{reg} \PY{o}{=} \PY{n}{LinearRegression}\PY{p}{(}\PY{p}{)}\PY{o}{.}\PY{n}{fit}\PY{p}{(}\PY{n}{x\PYZus{}A}\PY{o}{.}\PY{n}{reshape}\PY{p}{(}\PY{o}{\PYZhy{}}\PY{l+m+mi}{1}\PY{p}{,} \PY{l+m+mi}{1}\PY{p}{)}\PY{p}{,} \PY{n}{y}\PY{p}{)}
    \PY{n}{m} \PY{o}{=} \PY{n}{reg}\PY{o}{.}\PY{n}{coef\PYZus{}}
    \PY{n}{c} \PY{o}{=} \PY{n}{reg}\PY{o}{.}\PY{n}{intercept\PYZus{}}

    \PY{n}{y\PYZus{}pred} \PY{o}{=} \PY{n}{m}\PY{o}{*}\PY{n}{x\PYZus{}A} \PY{o}{+} \PY{n}{c}

    \PY{n}{plt}\PY{o}{.}\PY{n}{title}\PY{p}{(}\PY{l+s+s1}{\PYZsq{}}\PY{l+s+s1}{Refractive index vs mole fraction}\PY{l+s+s1}{\PYZsq{}}\PY{p}{)}
    \PY{n}{plt}\PY{o}{.}\PY{n}{xlabel}\PY{p}{(}\PY{l+s+s1}{\PYZsq{}}\PY{l+s+s1}{Mole fraction of A (\PYZdl{}X\PYZus{}A\PYZdl{})}\PY{l+s+s1}{\PYZsq{}}\PY{p}{)}
    \PY{n}{plt}\PY{o}{.}\PY{n}{ylabel}\PY{p}{(}\PY{l+s+s1}{\PYZsq{}}\PY{l+s+s1}{Refractive index (RI)}\PY{l+s+s1}{\PYZsq{}}\PY{p}{)}
    \PY{n}{plt}\PY{o}{.}\PY{n}{plot}\PY{p}{(}\PY{n}{x\PYZus{}A}\PY{p}{,} \PY{n}{y}\PY{p}{,} \PY{l+s+s1}{\PYZsq{}}\PY{l+s+s1}{o\PYZhy{}k}\PY{l+s+s1}{\PYZsq{}}\PY{p}{)}
    \PY{n}{plt}\PY{o}{.}\PY{n}{plot}\PY{p}{(}\PY{n}{x\PYZus{}A}\PY{p}{,} \PY{n}{y\PYZus{}pred}\PY{p}{,} \PY{l+s+s1}{\PYZsq{}}\PY{l+s+s1}{\PYZhy{}r}\PY{l+s+s1}{\PYZsq{}}\PY{p}{)}

    \PY{n+nb}{print}\PY{p}{(}\PY{l+s+sa}{f}\PY{l+s+s2}{\PYZdq{}}\PY{l+s+s2}{slope: }\PY{l+s+si}{\PYZob{}}\PY{n}{reg}\PY{o}{.}\PY{n}{coef\PYZus{}}\PY{l+s+si}{\PYZcb{}}\PY{l+s+s2}{, intercept: }\PY{l+s+si}{\PYZob{}}\PY{n}{reg}\PY{o}{.}\PY{n}{intercept\PYZus{}}\PY{l+s+si}{\PYZcb{}}\PY{l+s+s2}{\PYZdq{}}\PY{p}{)}
  \end{Verbatim}
\end{tcolorbox}

\begin{Verbatim}[commandchars=\\\{\}]
  slope: [-0.14296434], intercept: 1.5009384465620594
\end{Verbatim}

\begin{center}
  \adjustimage{max size={0.7\linewidth}{0.7\paperheight}}{images/output_6_1.png}
\end{center}
{ \hspace*{\fill} \\}

\begin{tcolorbox}[breakable, size=fbox, boxrule=1pt, pad at break*=1mm,colback=cellbackground, colframe=cellborder]
  \prompt{In}{incolor}{5}{\boxspacing}
  \begin{Verbatim}[commandchars=\\\{\}]
    \PY{n}{T} \PY{o}{=} \PY{p}{[}\PY{l+m+mf}{58.2}\PY{p}{,} \PY{l+m+mf}{59.9}\PY{p}{,} \PY{l+m+mf}{60.8}\PY{p}{,} \PY{l+m+mf}{63.0}\PY{p}{,} \PY{l+m+mf}{65.2}\PY{p}{,} \PY{l+m+mf}{68.1}\PY{p}{]}
    \PY{n}{T} \PY{o}{=} \PY{n}{np}\PY{o}{.}\PY{n}{array}\PY{p}{(}\PY{n}{T}\PY{p}{)}
    \PY{n}{T} \PY{o}{+}\PY{o}{=} \PY{l+m+mf}{273.15}

    \PY{n}{RI\PYZus{}vap\PYZus{}1} \PY{o}{=} \PY{p}{[}\PY{l+m+mf}{1.376541}\PY{p}{,} \PY{l+m+mf}{1.395104}\PY{p}{,} \PY{l+m+mf}{1.404485}\PY{p}{,} \PY{l+m+mf}{1.407679}\PY{p}{,} \PY{l+m+mf}{1.431531}\PY{p}{,} \PY{l+m+mf}{1.442809}\PY{p}{]}
    \PY{n}{RI\PYZus{}vap\PYZus{}2} \PY{o}{=} \PY{p}{[}\PY{l+m+mf}{1.375044}\PY{p}{,} \PY{l+m+mf}{1.396002}\PY{p}{,} \PY{l+m+mf}{1.399895}\PY{p}{,} \PY{l+m+mf}{1.414266}\PY{p}{,} \PY{l+m+mf}{1.431831}\PY{p}{,} \PY{l+m+mf}{1.445703}\PY{p}{]}
    \PY{n}{RI\PYZus{}vap\PYZus{}3} \PY{o}{=} \PY{p}{[}\PY{l+m+mf}{1.375344}\PY{p}{,} \PY{l+m+mf}{1.395803}\PY{p}{,} \PY{l+m+mf}{1.400693}\PY{p}{,} \PY{l+m+mf}{1.414665}\PY{p}{,} \PY{l+m+mf}{1.429635}\PY{p}{,} \PY{l+m+mf}{1.446701}\PY{p}{]}
    \PY{n}{RI\PYZus{}vap} \PY{o}{=} \PY{n}{np}\PY{o}{.}\PY{n}{mean}\PY{p}{(}\PY{p}{[}\PY{n}{RI\PYZus{}vap\PYZus{}1}\PY{p}{,} \PY{n}{RI\PYZus{}vap\PYZus{}2}\PY{p}{,} \PY{n}{RI\PYZus{}vap\PYZus{}3}\PY{p}{]}\PY{p}{,} \PY{n}{axis}\PY{o}{=}\PY{l+m+mi}{0}\PY{p}{)}
    \PY{n}{Y\PYZus{}A} \PY{o}{=} \PY{p}{(}\PY{n}{RI\PYZus{}vap} \PY{o}{\PYZhy{}} \PY{n}{c}\PY{p}{)} \PY{o}{/} \PY{n}{m}
    \PY{n}{Y\PYZus{}B} \PY{o}{=} \PY{l+m+mi}{1} \PY{o}{\PYZhy{}} \PY{n}{Y\PYZus{}A}

    \PY{n}{RI\PYZus{}liq\PYZus{}1} \PY{o}{=} \PY{p}{[}\PY{l+m+mf}{1.382729}\PY{p}{,} \PY{l+m+mf}{1.403787}\PY{p}{,} \PY{l+m+mf}{1.434326}\PY{p}{,} \PY{l+m+mf}{1.444006}\PY{p}{,} \PY{l+m+mf}{1.449495}\PY{p}{,} \PY{l+m+mf}{1.473048}\PY{p}{]}
    \PY{n}{RI\PYZus{}liq\PYZus{}2} \PY{o}{=} \PY{p}{[}\PY{l+m+mf}{1.381831}\PY{p}{,} \PY{l+m+mf}{1.407579}\PY{p}{,} \PY{l+m+mf}{1.436621}\PY{p}{,} \PY{l+m+mf}{1.441611}\PY{p}{,} \PY{l+m+mf}{1.450194}\PY{p}{,} \PY{l+m+mf}{1.473747}\PY{p}{]}
    \PY{n}{RI\PYZus{}liq\PYZus{}3} \PY{o}{=} \PY{p}{[}\PY{l+m+mf}{1.385324}\PY{p}{,} \PY{l+m+mf}{1.408178}\PY{p}{,} \PY{l+m+mf}{1.436222}\PY{p}{,} \PY{l+m+mf}{1.443507}\PY{p}{,} \PY{l+m+mf}{1.450793}\PY{p}{,} \PY{l+m+mf}{1.472649}\PY{p}{]}
    \PY{n}{RI\PYZus{}liq} \PY{o}{=} \PY{n}{np}\PY{o}{.}\PY{n}{mean}\PY{p}{(}\PY{p}{[}\PY{n}{RI\PYZus{}liq\PYZus{}1}\PY{p}{,} \PY{n}{RI\PYZus{}liq\PYZus{}2}\PY{p}{,} \PY{n}{RI\PYZus{}liq\PYZus{}3}\PY{p}{]}\PY{p}{,} \PY{n}{axis}\PY{o}{=}\PY{l+m+mi}{0}\PY{p}{)}
    \PY{n}{X\PYZus{}A} \PY{o}{=} \PY{p}{(}\PY{n}{RI\PYZus{}liq} \PY{o}{\PYZhy{}} \PY{n}{c}\PY{p}{)} \PY{o}{/} \PY{n}{m}
    \PY{n}{X\PYZus{}B} \PY{o}{=} \PY{l+m+mi}{1} \PY{o}{\PYZhy{}} \PY{n}{X\PYZus{}A}

    \PY{n+nb}{print}\PY{p}{(}\PY{l+s+s2}{\PYZdq{}}\PY{l+s+s2}{Average mole fraction of vapour A (at various temp):}\PY{l+s+s2}{\PYZdq{}}\PY{p}{,} \PY{n}{Y\PYZus{}A}\PY{p}{)}
    \PY{n+nb}{print}\PY{p}{(}\PY{l+s+s2}{\PYZdq{}}\PY{l+s+s2}{Average mole fraction of liquid A (at various temp):}\PY{l+s+s2}{\PYZdq{}}\PY{p}{,} \PY{n}{X\PYZus{}A}\PY{p}{)}
  \end{Verbatim}
\end{tcolorbox}

\begin{Verbatim}[commandchars=\\\{\}]
  Average mole fraction of vapour A (at various temp): [0.87641051 0.73656211
  0.69421122 0.62068006 0.48920905 0.3907789 ]
  Average mole fraction of liquid A (at various temp): [0.822889   0.66047087
  0.45616584 0.40497592 0.35517795 0.19438727]
\end{Verbatim}

\hypertarget{calculate-p_sat}{%
  \subsection{\texorpdfstring{Calculate
      \(P_{sat}\)}{Calculate P\_\{sat\}}}\label{calculate-p_sat}}

\begin{itemize}
  \tightlist
  \item
        Antoine Equation: \(log_{10}(P) = A − \frac{B}{T + C}\)
\end{itemize}

\hypertarget{acetone}{%
  \subsubsection{Acetone}\label{acetone}}

\begin{itemize}
  \tightlist
  \item
        A = 4.42448
  \item
        B = 1312.253
  \item
        C = -32.445
  \item
        reference: \emph{Ambrose, Sprake, et al., 1974}
\end{itemize}

\hypertarget{benzene}{%
  \subsubsection{Benzene}\label{benzene}}

\begin{itemize}
  \tightlist
  \item
        A = 4.725
  \item
        B = 1660.652
  \item
        C = -1.461
  \item
        reference: \emph{Eon, Pommier, et al., 1971}
\end{itemize}

\begin{tcolorbox}[breakable, size=fbox, boxrule=1pt, pad at break*=1mm,colback=cellbackground, colframe=cellborder]
  \prompt{In}{incolor}{6}{\boxspacing}
  \begin{Verbatim}[commandchars=\\\{\}]
    \PY{n}{log10\PYZus{}Psat\PYZus{}A} \PY{o}{=} \PY{l+m+mf}{4.424} \PY{o}{\PYZhy{}} \PY{l+m+mf}{1312.25} \PY{o}{/} \PY{p}{(}\PY{n}{T} \PY{o}{\PYZhy{}} \PY{l+m+mf}{32.44}\PY{p}{)}
    \PY{n}{Psat\PYZus{}A} \PY{o}{=} \PY{n}{np}\PY{o}{.}\PY{n}{power}\PY{p}{(}\PY{l+m+mi}{10}\PY{p}{,} \PY{n}{log10\PYZus{}Psat\PYZus{}A}\PY{p}{)}
    \PY{n+nb}{print}\PY{p}{(}\PY{l+s+s2}{\PYZdq{}}\PY{l+s+s2}{Saturation pressure of Acetone:}\PY{l+s+s2}{\PYZdq{}}\PY{p}{,} \PY{n}{Psat\PYZus{}A}\PY{p}{)}

    \PY{n}{log10\PYZus{}Psat\PYZus{}B} \PY{o}{=} \PY{l+m+mf}{4.725} \PY{o}{\PYZhy{}} \PY{l+m+mf}{1660.65} \PY{o}{/} \PY{p}{(}\PY{n}{T} \PY{o}{\PYZhy{}} \PY{l+m+mf}{1.46}\PY{p}{)}
    \PY{n}{Psat\PYZus{}B} \PY{o}{=} \PY{n}{np}\PY{o}{.}\PY{n}{power}\PY{p}{(}\PY{l+m+mi}{10}\PY{p}{,} \PY{n}{log10\PYZus{}Psat\PYZus{}B}\PY{p}{)}
    \PY{n+nb}{print}\PY{p}{(}\PY{l+s+s2}{\PYZdq{}}\PY{l+s+s2}{Saturation pressure of Benzene:}\PY{l+s+s2}{\PYZdq{}}\PY{p}{,} \PY{n}{Psat\PYZus{}B}\PY{p}{)}
  \end{Verbatim}
\end{tcolorbox}

\begin{Verbatim}[commandchars=\\\{\}]
  Saturation pressure of Acetone: [1.08114159 1.14474676 1.17961343 1.26842971
  1.36250982 1.49493862]
  Saturation pressure of Benzene: [0.49096359 0.52102362 0.5375436  0.57975479
  0.62466361 0.68820318]
\end{Verbatim}

\begin{tcolorbox}[breakable, size=fbox, boxrule=1pt, pad at break*=1mm,colback=cellbackground, colframe=cellborder]
  \prompt{In}{incolor}{7}{\boxspacing}
  \begin{Verbatim}[commandchars=\\\{\}]
    \PY{n}{gamma\PYZus{}A} \PY{o}{=} \PY{p}{(}\PY{n}{Y\PYZus{}A} \PY{o}{*} \PY{l+m+mi}{1}\PY{p}{)} \PY{o}{/} \PY{p}{(}\PY{n}{Psat\PYZus{}A} \PY{o}{*} \PY{n}{X\PYZus{}A}\PY{p}{)}
    \PY{n}{ln\PYZus{}gamma\PYZus{}A} \PY{o}{=} \PY{n}{np}\PY{o}{.}\PY{n}{log}\PY{p}{(}\PY{n}{gamma\PYZus{}A}\PY{p}{)}
    \PY{n+nb}{print}\PY{p}{(}\PY{l+s+s2}{\PYZdq{}}\PY{l+s+s2}{ln(gamma\PYZus{}A):}\PY{l+s+s2}{\PYZdq{}}\PY{p}{,} \PY{n}{ln\PYZus{}gamma\PYZus{}A}\PY{p}{)}

    \PY{c+c1}{\PYZsh{} Psat\PYZus{}B *= 1.3}
    \PY{n}{gamma\PYZus{}B} \PY{o}{=} \PY{p}{(}\PY{n}{Y\PYZus{}B} \PY{o}{*} \PY{l+m+mi}{1}\PY{p}{)} \PY{o}{/} \PY{p}{(}\PY{n}{Psat\PYZus{}B} \PY{o}{*} \PY{n}{X\PYZus{}B}\PY{p}{)}
    \PY{n}{ln\PYZus{}gamma\PYZus{}B} \PY{o}{=} \PY{n}{np}\PY{o}{.}\PY{n}{log}\PY{p}{(}\PY{n}{gamma\PYZus{}B}\PY{p}{)}
    \PY{n+nb}{print}\PY{p}{(}\PY{l+s+s2}{\PYZdq{}}\PY{l+s+s2}{ln(gamma\PYZus{}B):}\PY{l+s+s2}{\PYZdq{}}\PY{p}{,} \PY{n}{ln\PYZus{}gamma\PYZus{}B}\PY{p}{)}
  \end{Verbatim}
\end{tcolorbox}

\begin{Verbatim}[commandchars=\\\{\}]
  ln(gamma\_A): [-0.01500423 -0.0261429   0.25473305  0.18920846  0.01084251
  0.2962044 ]
  ln(gamma\_B): [0.35157418 0.39821779 0.04499565 0.09492817 0.23752803 0.09424926]
\end{Verbatim}

\begin{tcolorbox}[breakable, size=fbox, boxrule=1pt, pad at break*=1mm,colback=cellbackground, colframe=cellborder]
  \prompt{In}{incolor}{8}{\boxspacing}
  \begin{Verbatim}[commandchars=\\\{\}]
    \PY{n}{plt}\PY{o}{.}\PY{n}{axhline}\PY{p}{(}\PY{n}{y}\PY{o}{=}\PY{l+m+mi}{0}\PY{p}{,} \PY{n}{color}\PY{o}{=}\PY{l+s+s1}{\PYZsq{}}\PY{l+s+s1}{k}\PY{l+s+s1}{\PYZsq{}}\PY{p}{)}
    \PY{n}{plt}\PY{o}{.}\PY{n}{plot}\PY{p}{(}\PY{n}{X\PYZus{}A}\PY{p}{,} \PY{n}{ln\PYZus{}gamma\PYZus{}A}\PY{p}{,} \PY{l+s+s1}{\PYZsq{}}\PY{l+s+s1}{o\PYZhy{}b}\PY{l+s+s1}{\PYZsq{}}\PY{p}{,} \PY{n}{label}\PY{o}{=}\PY{l+s+s2}{\PYZdq{}}\PY{l+s+s2}{Acetone}\PY{l+s+s2}{\PYZdq{}}\PY{p}{)}
    \PY{n}{plt}\PY{o}{.}\PY{n}{plot}\PY{p}{(}\PY{n}{X\PYZus{}A}\PY{p}{,} \PY{n}{ln\PYZus{}gamma\PYZus{}B}\PY{p}{,} \PY{l+s+s1}{\PYZsq{}}\PY{l+s+s1}{o\PYZhy{}g}\PY{l+s+s1}{\PYZsq{}}\PY{p}{,} \PY{n}{label}\PY{o}{=}\PY{l+s+s2}{\PYZdq{}}\PY{l+s+s2}{Benzene}\PY{l+s+s2}{\PYZdq{}}\PY{p}{)}
    \PY{n}{plt}\PY{o}{.}\PY{n}{legend}\PY{p}{(}\PY{p}{)}
    \PY{n}{plt}\PY{o}{.}\PY{n}{xlabel}\PY{p}{(}\PY{l+s+s2}{\PYZdq{}}\PY{l+s+s2}{Mole fraction of Acetone}\PY{l+s+s2}{\PYZdq{}}\PY{p}{)}
    \PY{n}{plt}\PY{o}{.}\PY{n}{ylabel}\PY{p}{(}\PY{l+s+s2}{\PYZdq{}}\PY{l+s+s2}{ln(\PYZdl{}}\PY{l+s+s2}{\PYZbs{}}\PY{l+s+s2}{gamma\PYZdl{})}\PY{l+s+s2}{\PYZdq{}}\PY{p}{)}
    \PY{n}{plt}\PY{o}{.}\PY{n}{title}\PY{p}{(}\PY{l+s+s2}{\PYZdq{}}\PY{l+s+s2}{Mole fraction of Acetone vs ln(\PYZdl{}}\PY{l+s+s2}{\PYZbs{}}\PY{l+s+s2}{gamma\PYZdl{})}\PY{l+s+s2}{\PYZdq{}}\PY{p}{)}
    \PY{n}{plt}\PY{o}{.}\PY{n}{show}\PY{p}{(}\PY{p}{)}
  \end{Verbatim}
\end{tcolorbox}

\begin{center}
  \adjustimage{max size={0.7\linewidth}{0.7\paperheight}}{images/output_11_0.png}
\end{center}
{ \hspace*{\fill} \\}

\begin{tcolorbox}[breakable, size=fbox, boxrule=1pt, pad at break*=1mm,colback=cellbackground, colframe=cellborder]
  \prompt{In}{incolor}{9}{\boxspacing}
  \begin{Verbatim}[commandchars=\\\{\}]
    \PY{n}{delta\PYZus{}ln\PYZus{}gamma} \PY{o}{=} \PY{n}{ln\PYZus{}gamma\PYZus{}A}\PY{o}{\PYZhy{}}\PY{n}{ln\PYZus{}gamma\PYZus{}B}

    \PY{n}{plt}\PY{o}{.}\PY{n}{axhline}\PY{p}{(}\PY{n}{y}\PY{o}{=}\PY{l+m+mi}{0}\PY{p}{,} \PY{n}{color}\PY{o}{=}\PY{l+s+s1}{\PYZsq{}}\PY{l+s+s1}{k}\PY{l+s+s1}{\PYZsq{}}\PY{p}{)}
    \PY{n}{plt}\PY{o}{.}\PY{n}{plot}\PY{p}{(}\PY{n}{X\PYZus{}A}\PY{p}{,} \PY{n}{delta\PYZus{}ln\PYZus{}gamma}\PY{p}{,} \PY{l+s+s1}{\PYZsq{}}\PY{l+s+s1}{o\PYZhy{}k}\PY{l+s+s1}{\PYZsq{}}\PY{p}{)}
    \PY{n}{plt}\PY{o}{.}\PY{n}{xlabel}\PY{p}{(}\PY{l+s+s2}{\PYZdq{}}\PY{l+s+s2}{Mole fraction of Acetone}\PY{l+s+s2}{\PYZdq{}}\PY{p}{)}
    \PY{n}{plt}\PY{o}{.}\PY{n}{ylabel}\PY{p}{(}\PY{l+s+s2}{\PYZdq{}}\PY{l+s+s2}{ln(\PYZdl{}}\PY{l+s+s2}{\PYZbs{}}\PY{l+s+s2}{gamma\PYZus{}A\PYZdl{}) \PYZhy{} ln(\PYZdl{}}\PY{l+s+s2}{\PYZbs{}}\PY{l+s+s2}{gamma\PYZus{}B\PYZdl{})}\PY{l+s+s2}{\PYZdq{}}\PY{p}{)}
    \PY{n}{plt}\PY{o}{.}\PY{n}{title}\PY{p}{(}\PY{l+s+s2}{\PYZdq{}}\PY{l+s+s2}{Mole fraction of Acetone vs \PYZdl{}}\PY{l+s+s2}{\PYZbs{}}\PY{l+s+s2}{Delta\PYZdl{} ln(\PYZdl{}}\PY{l+s+s2}{\PYZbs{}}\PY{l+s+s2}{gamma\PYZdl{})}\PY{l+s+s2}{\PYZdq{}}\PY{p}{)}
    \PY{n}{plt}\PY{o}{.}\PY{n}{fill\PYZus{}between}\PY{p}{(}\PY{n}{X\PYZus{}A}\PY{p}{,} \PY{l+m+mi}{0}\PY{p}{,} \PY{n}{delta\PYZus{}ln\PYZus{}gamma}\PY{p}{)}
    \PY{n}{plt}\PY{o}{.}\PY{n}{show}\PY{p}{(}\PY{p}{)}
  \end{Verbatim}
\end{tcolorbox}

\begin{center}
  \adjustimage{max size={0.7\linewidth}{0.7\paperheight}}{images/output_12_0.png}
\end{center}
{ \hspace*{\fill} \\}

\begin{tcolorbox}[breakable, size=fbox, boxrule=1pt, pad at break*=1mm,colback=cellbackground, colframe=cellborder]
  \prompt{In}{incolor}{10}{\boxspacing}
  \begin{Verbatim}[commandchars=\\\{\}]
    \PY{n}{spline} \PY{o}{=} \PY{n}{interpolate}\PY{o}{.}\PY{n}{interp1d}\PY{p}{(}\PY{n}{X\PYZus{}A}\PY{p}{,} \PY{n}{delta\PYZus{}ln\PYZus{}gamma}\PY{p}{,} \PY{n}{kind}\PY{o}{=}\PY{l+s+s1}{\PYZsq{}}\PY{l+s+s1}{cubic}\PY{l+s+s1}{\PYZsq{}}\PY{p}{)}

    \PY{n}{x\PYZus{}inter} \PY{o}{=} \PY{n}{np}\PY{o}{.}\PY{n}{linspace}\PY{p}{(}\PY{l+m+mf}{0.2}\PY{p}{,} \PY{l+m+mf}{0.8}\PY{p}{,} \PY{l+m+mi}{300}\PY{p}{)}
    \PY{n}{y\PYZus{}inter} \PY{o}{=} \PY{n}{spline}\PY{p}{(}\PY{n}{x\PYZus{}inter}\PY{p}{)}

    \PY{n}{plt}\PY{o}{.}\PY{n}{axhline}\PY{p}{(}\PY{n}{y}\PY{o}{=}\PY{l+m+mi}{0}\PY{p}{,} \PY{n}{color}\PY{o}{=}\PY{l+s+s1}{\PYZsq{}}\PY{l+s+s1}{k}\PY{l+s+s1}{\PYZsq{}}\PY{p}{)}
    \PY{n}{plt}\PY{o}{.}\PY{n}{plot}\PY{p}{(}\PY{n}{x\PYZus{}inter}\PY{p}{,} \PY{n}{y\PYZus{}inter}\PY{p}{,} \PY{l+s+s1}{\PYZsq{}}\PY{l+s+s1}{\PYZhy{}b}\PY{l+s+s1}{\PYZsq{}}\PY{p}{)}
    \PY{n}{plt}\PY{o}{.}\PY{n}{title}\PY{p}{(}\PY{l+s+s2}{\PYZdq{}}\PY{l+s+s2}{Mole fraction of Acetone vs \PYZdl{}}\PY{l+s+s2}{\PYZbs{}}\PY{l+s+s2}{Delta\PYZdl{} ln(\PYZdl{}}\PY{l+s+s2}{\PYZbs{}}\PY{l+s+s2}{gamma\PYZdl{}) after spline interpolation}\PY{l+s+s2}{\PYZdq{}}\PY{p}{)}
    \PY{n}{plt}\PY{o}{.}\PY{n}{xlabel}\PY{p}{(}\PY{l+s+s2}{\PYZdq{}}\PY{l+s+s2}{Mole fraction of Acetone}\PY{l+s+s2}{\PYZdq{}}\PY{p}{)}
    \PY{n}{plt}\PY{o}{.}\PY{n}{ylabel}\PY{p}{(}\PY{l+s+s2}{\PYZdq{}}\PY{l+s+s2}{ln(\PYZdl{}}\PY{l+s+s2}{\PYZbs{}}\PY{l+s+s2}{gamma\PYZus{}A\PYZdl{}) \PYZhy{} ln(\PYZdl{}}\PY{l+s+s2}{\PYZbs{}}\PY{l+s+s2}{gamma\PYZus{}B\PYZdl{})}\PY{l+s+s2}{\PYZdq{}}\PY{p}{)}

    \PY{n}{area} \PY{o}{=} \PY{l+m+mi}{0}
    \PY{k}{for} \PY{n}{i} \PY{o+ow}{in} \PY{n+nb}{range}\PY{p}{(}\PY{n+nb}{len}\PY{p}{(}\PY{n}{x\PYZus{}inter}\PY{p}{)}\PY{o}{\PYZhy{}}\PY{l+m+mi}{1}\PY{p}{)}\PY{p}{:}
    \PY{n}{area} \PY{o}{+}\PY{o}{=} \PY{p}{(}\PY{n}{y\PYZus{}inter}\PY{p}{[}\PY{n}{i}\PY{p}{]}\PY{p}{)}\PY{o}{*}\PY{p}{(}\PY{n}{x\PYZus{}inter}\PY{p}{[}\PY{n}{i}\PY{o}{+}\PY{l+m+mi}{1}\PY{p}{]}\PY{o}{\PYZhy{}}\PY{n}{x\PYZus{}inter}\PY{p}{[}\PY{n}{i}\PY{p}{]}\PY{p}{)}

    \PY{n+nb}{print}\PY{p}{(}\PY{l+s+s2}{\PYZdq{}}\PY{l+s+s2}{Area:}\PY{l+s+s2}{\PYZdq{}}\PY{p}{,} \PY{n}{area}\PY{p}{)}
  \end{Verbatim}
\end{tcolorbox}

\begin{Verbatim}[commandchars=\\\{\}]
  Area: -0.14447233309865778
\end{Verbatim}

\begin{center}
  \adjustimage{max size={0.7\linewidth}{0.7\paperheight}}{images/output_13_1.png}
\end{center}
{ \hspace*{\fill} \\}

\begin{tcolorbox}[breakable, size=fbox, boxrule=1pt, pad at break*=1mm,colback=cellbackground, colframe=cellborder]
  \prompt{In}{incolor}{11}{\boxspacing}
  \begin{Verbatim}[commandchars=\\\{\}]
    \PY{n}{alpha} \PY{o}{=} \PY{p}{(}\PY{n}{X\PYZus{}B} \PY{o}{*} \PY{n}{ln\PYZus{}gamma\PYZus{}B}\PY{p}{)} \PY{o}{/} \PY{p}{(}\PY{n}{X\PYZus{}A} \PY{o}{*} \PY{n}{ln\PYZus{}gamma\PYZus{}A}\PY{p}{)}
    \PY{n}{a} \PY{o}{=} \PY{n}{ln\PYZus{}gamma\PYZus{}A} \PY{o}{*} \PY{p}{(}\PY{l+m+mi}{1} \PY{o}{+} \PY{n}{alpha}\PY{p}{)}\PY{o}{*}\PY{o}{*}\PY{l+m+mi}{2}
    \PY{n}{b} \PY{o}{=} \PY{n}{ln\PYZus{}gamma\PYZus{}B} \PY{o}{*} \PY{p}{(}\PY{l+m+mi}{1} \PY{o}{+} \PY{l+m+mf}{1.0}\PY{o}{/}\PY{n}{alpha}\PY{p}{)}\PY{o}{*}\PY{o}{*}\PY{l+m+mi}{2}

    \PY{n+nb}{print}\PY{p}{(}\PY{l+s+sa}{f}\PY{l+s+s2}{\PYZdq{}}\PY{l+s+s2}{Van Laar coeffecients}\PY{l+s+s2}{\PYZdq{}}\PY{p}{)}
    \PY{n+nb}{print}\PY{p}{(}\PY{l+s+sa}{f}\PY{l+s+s2}{\PYZdq{}}\PY{l+s+s2}{a: }\PY{l+s+si}{\PYZob{}}\PY{n}{a}\PY{l+s+si}{\PYZcb{}}\PY{l+s+s2}{\PYZdq{}}\PY{p}{)}
    \PY{n+nb}{print}\PY{p}{(}\PY{l+s+sa}{f}\PY{l+s+s2}{\PYZdq{}}\PY{l+s+s2}{b: }\PY{l+s+si}{\PYZob{}}\PY{n}{b}\PY{l+s+si}{\PYZcb{}}\PY{l+s+s2}{\PYZdq{}}\PY{p}{)}
  \end{Verbatim}
\end{tcolorbox}

\begin{Verbatim}[commandchars=\\\{\}]
  Van Laar coeffecients
  a: [-0.24528301 -1.21972077  0.37331584  0.57097702 18.02423931  1.5925    ]
  b: [0.22597255 0.30300295 1.48697511 0.52717326 0.24962263 0.29139089]
\end{Verbatim}

\begin{tcolorbox}[breakable, size=fbox, boxrule=1pt, pad at break*=1mm,colback=cellbackground, colframe=cellborder]
  \prompt{In}{incolor}{12}{\boxspacing}
  \begin{Verbatim}[commandchars=\\\{\}]
    \PY{n}{residual} \PY{o}{=} \PY{n}{X\PYZus{}A} \PY{o}{*} \PY{n}{ln\PYZus{}gamma\PYZus{}A} \PY{o}{+} \PY{n}{X\PYZus{}B} \PY{o}{*} \PY{n}{ln\PYZus{}gamma\PYZus{}B}
    \PY{n+nb}{print}\PY{p}{(}\PY{l+s+s1}{\PYZsq{}}\PY{l+s+s1}{Residual: }\PY{l+s+s1}{\PYZsq{}}\PY{p}{,} \PY{n}{residual}\PY{p}{)}
  \end{Verbatim}
\end{tcolorbox}

\begin{Verbatim}[commandchars=\\\{\}]
  Residual:  [0.04992084 0.11793991 0.14067069 0.13310942 0.15701433 0.13350677]
\end{Verbatim}

\begin{tcolorbox}[breakable, size=fbox, boxrule=1pt, pad at break*=1mm,colback=cellbackground, colframe=cellborder]
  \prompt{In}{incolor}{13}{\boxspacing}
  \begin{Verbatim}[commandchars=\\\{\}]
    \PY{n}{K\PYZus{}A} \PY{o}{=} \PY{n}{Y\PYZus{}A} \PY{o}{/} \PY{n}{X\PYZus{}A}
    \PY{n+nb}{print} \PY{p}{(}\PY{l+s+s2}{\PYZdq{}}\PY{l+s+s2}{K\PYZus{}A: }\PY{l+s+s2}{\PYZdq{}} \PY{p}{,} \PY{n}{K\PYZus{}A}\PY{p}{)}
  \end{Verbatim}
\end{tcolorbox}

\begin{Verbatim}[commandchars=\\\{\}]
  K\_A:  [1.06504098 1.11520756 1.52183956 1.5326345  1.37736322 2.01031122]
\end{Verbatim}


% Add a bibliography block to the postdoc



\end{document}

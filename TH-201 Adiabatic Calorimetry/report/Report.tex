\documentclass[11pt]{article}

    \usepackage[breakable]{tcolorbox}
    \usepackage{parskip} % Stop auto-indenting (to mimic markdown behaviour)
    
    \usepackage{iftex}
    \ifPDFTeX
      \usepackage[T1]{fontenc}
      \usepackage{mathpazo}
    \else
      \usepackage{fontspec}
    \fi

    % Basic figure setup, for now with no caption control since it's done
    % automatically by Pandoc (which extracts ![](path) syntax from Markdown).
    \usepackage{graphicx}
    % Maintain compatibility with old templates. Remove in nbconvert 6.0
    \let\Oldincludegraphics\includegraphics
    % Ensure that by default, figures have no caption (until we provide a
    % proper Figure object with a Caption API and a way to capture that
    % in the conversion process - todo).
    \usepackage{caption}
    \DeclareCaptionFormat{nocaption}{}
    \captionsetup{format=nocaption,aboveskip=0pt,belowskip=0pt}

    \usepackage{float}
    \floatplacement{figure}{H} % forces figures to be placed at the correct location
    \usepackage{xcolor} % Allow colors to be defined
    \usepackage{enumerate} % Needed for markdown enumerations to work
    \usepackage{geometry} % Used to adjust the document margins
    \usepackage{amsmath} % Equations
    \usepackage{amssymb} % Equations
    \usepackage{textcomp} % defines textquotesingle
    % Hack from http://tex.stackexchange.com/a/47451/13684:
    \AtBeginDocument{%
        \def\PYZsq{\textquotesingle}% Upright quotes in Pygmentized code
    }
    \usepackage{upquote} % Upright quotes for verbatim code
    \usepackage{eurosym} % defines \euro
    \usepackage[mathletters]{ucs} % Extended unicode (utf-8) support
    \usepackage{fancyvrb} % verbatim replacement that allows latex
    \usepackage{grffile} % extends the file name processing of package graphics 
                        % to support a larger range
    \makeatletter % fix for old versions of grffile with XeLaTeX
    \@ifpackagelater{grffile}{2019/11/01}
    {
      % Do nothing on new versions
    }
    {
      \def\Gread@@xetex#1{%
        \IfFileExists{"\Gin@base".bb}%
        {\Gread@eps{\Gin@base.bb}}%
        {\Gread@@xetex@aux#1}%
      }
    }
    \makeatother
    \usepackage[Export]{adjustbox} % Used to constrain images to a maximum size
    \adjustboxset{max size={0.9\linewidth}{0.9\paperheight}}

    % The hyperref package gives us a pdf with properly built
    % internal navigation ('pdf bookmarks' for the table of contents,
    % internal cross-reference links, web links for URLs, etc.)
    \usepackage{hyperref}
    % The default LaTeX title has an obnoxious amount of whitespace. By default,
    % titling removes some of it. It also provides customization options.
    \usepackage{titling}
    \usepackage{longtable} % longtable support required by pandoc >1.10
    \usepackage{booktabs}  % table support for pandoc > 1.12.2
    \usepackage[inline]{enumitem} % IRkernel/repr support (it uses the enumerate* environment)
    \usepackage[normalem]{ulem} % ulem is needed to support strikethroughs (\sout)
                                % normalem makes italics be italics, not underlines
    \usepackage{mathrsfs}
    

    
    % Colors for the hyperref package
    \definecolor{urlcolor}{rgb}{0,.145,.698}
    \definecolor{linkcolor}{rgb}{.71,0.21,0.01}
    \definecolor{citecolor}{rgb}{.12,.54,.11}

    % ANSI colors
    \definecolor{ansi-black}{HTML}{3E424D}
    \definecolor{ansi-black-intense}{HTML}{282C36}
    \definecolor{ansi-red}{HTML}{E75C58}
    \definecolor{ansi-red-intense}{HTML}{B22B31}
    \definecolor{ansi-green}{HTML}{00A250}
    \definecolor{ansi-green-intense}{HTML}{007427}
    \definecolor{ansi-yellow}{HTML}{DDB62B}
    \definecolor{ansi-yellow-intense}{HTML}{B27D12}
    \definecolor{ansi-blue}{HTML}{208FFB}
    \definecolor{ansi-blue-intense}{HTML}{0065CA}
    \definecolor{ansi-magenta}{HTML}{D160C4}
    \definecolor{ansi-magenta-intense}{HTML}{A03196}
    \definecolor{ansi-cyan}{HTML}{60C6C8}
    \definecolor{ansi-cyan-intense}{HTML}{258F8F}
    \definecolor{ansi-white}{HTML}{C5C1B4}
    \definecolor{ansi-white-intense}{HTML}{A1A6B2}
    \definecolor{ansi-default-inverse-fg}{HTML}{FFFFFF}
    \definecolor{ansi-default-inverse-bg}{HTML}{000000}

    % common color for the border for error outputs.
    \definecolor{outerrorbackground}{HTML}{FFDFDF}

    % commands and environments needed by pandoc snippets
    % extracted from the output of `pandoc -s`
    \providecommand{\tightlist}{%
      \setlength{\itemsep}{0pt}\setlength{\parskip}{0pt}}
    \DefineVerbatimEnvironment{Highlighting}{Verbatim}{commandchars=\\\{\}}
    % Add ',fontsize=\small' for more characters per line
    \newenvironment{Shaded}{}{}
    \newcommand{\KeywordTok}[1]{\textcolor[rgb]{0.00,0.44,0.13}{\textbf{{#1}}}}
    \newcommand{\DataTypeTok}[1]{\textcolor[rgb]{0.56,0.13,0.00}{{#1}}}
    \newcommand{\DecValTok}[1]{\textcolor[rgb]{0.25,0.63,0.44}{{#1}}}
    \newcommand{\BaseNTok}[1]{\textcolor[rgb]{0.25,0.63,0.44}{{#1}}}
    \newcommand{\FloatTok}[1]{\textcolor[rgb]{0.25,0.63,0.44}{{#1}}}
    \newcommand{\CharTok}[1]{\textcolor[rgb]{0.25,0.44,0.63}{{#1}}}
    \newcommand{\StringTok}[1]{\textcolor[rgb]{0.25,0.44,0.63}{{#1}}}
    \newcommand{\CommentTok}[1]{\textcolor[rgb]{0.38,0.63,0.69}{\textit{{#1}}}}
    \newcommand{\OtherTok}[1]{\textcolor[rgb]{0.00,0.44,0.13}{{#1}}}
    \newcommand{\AlertTok}[1]{\textcolor[rgb]{1.00,0.00,0.00}{\textbf{{#1}}}}
    \newcommand{\FunctionTok}[1]{\textcolor[rgb]{0.02,0.16,0.49}{{#1}}}
    \newcommand{\RegionMarkerTok}[1]{{#1}}
    \newcommand{\ErrorTok}[1]{\textcolor[rgb]{1.00,0.00,0.00}{\textbf{{#1}}}}
    \newcommand{\NormalTok}[1]{{#1}}
    
    % Additional commands for more recent versions of Pandoc
    \newcommand{\ConstantTok}[1]{\textcolor[rgb]{0.53,0.00,0.00}{{#1}}}
    \newcommand{\SpecialCharTok}[1]{\textcolor[rgb]{0.25,0.44,0.63}{{#1}}}
    \newcommand{\VerbatimStringTok}[1]{\textcolor[rgb]{0.25,0.44,0.63}{{#1}}}
    \newcommand{\SpecialStringTok}[1]{\textcolor[rgb]{0.73,0.40,0.53}{{#1}}}
    \newcommand{\ImportTok}[1]{{#1}}
    \newcommand{\DocumentationTok}[1]{\textcolor[rgb]{0.73,0.13,0.13}{\textit{{#1}}}}
    \newcommand{\AnnotationTok}[1]{\textcolor[rgb]{0.38,0.63,0.69}{\textbf{\textit{{#1}}}}}
    \newcommand{\CommentVarTok}[1]{\textcolor[rgb]{0.38,0.63,0.69}{\textbf{\textit{{#1}}}}}
    \newcommand{\VariableTok}[1]{\textcolor[rgb]{0.10,0.09,0.49}{{#1}}}
    \newcommand{\ControlFlowTok}[1]{\textcolor[rgb]{0.00,0.44,0.13}{\textbf{{#1}}}}
    \newcommand{\OperatorTok}[1]{\textcolor[rgb]{0.40,0.40,0.40}{{#1}}}
    \newcommand{\BuiltInTok}[1]{{#1}}
    \newcommand{\ExtensionTok}[1]{{#1}}
    \newcommand{\PreprocessorTok}[1]{\textcolor[rgb]{0.74,0.48,0.00}{{#1}}}
    \newcommand{\AttributeTok}[1]{\textcolor[rgb]{0.49,0.56,0.16}{{#1}}}
    \newcommand{\InformationTok}[1]{\textcolor[rgb]{0.38,0.63,0.69}{\textbf{\textit{{#1}}}}}
    \newcommand{\WarningTok}[1]{\textcolor[rgb]{0.38,0.63,0.69}{\textbf{\textit{{#1}}}}}
    
    
    % Define a nice break command that doesn't care if a line doesn't already
    % exist.
    \def\br{\hspace*{\fill} \\* }
    % Math Jax compatibility definitions
    \def\gt{>}
    \def\lt{<}
    \let\Oldtex\TeX
    \let\Oldlatex\LaTeX
    \renewcommand{\TeX}{\textrm{\Oldtex}}
    \renewcommand{\LaTeX}{\textrm{\Oldlatex}}
    % Document parameters
    % Document title
    \title{Report}
    
    
    
    
    
% Pygments definitions
\makeatletter
\def\PY@reset{\let\PY@it=\relax \let\PY@bf=\relax%
    \let\PY@ul=\relax \let\PY@tc=\relax%
    \let\PY@bc=\relax \let\PY@ff=\relax}
\def\PY@tok#1{\csname PY@tok@#1\endcsname}
\def\PY@toks#1+{\ifx\relax#1\empty\else%
    \PY@tok{#1}\expandafter\PY@toks\fi}
\def\PY@do#1{\PY@bc{\PY@tc{\PY@ul{%
    \PY@it{\PY@bf{\PY@ff{#1}}}}}}}
\def\PY#1#2{\PY@reset\PY@toks#1+\relax+\PY@do{#2}}

\@namedef{PY@tok@w}{\def\PY@tc##1{\textcolor[rgb]{0.73,0.73,0.73}{##1}}}
\@namedef{PY@tok@c}{\let\PY@it=\textit\def\PY@tc##1{\textcolor[rgb]{0.25,0.50,0.50}{##1}}}
\@namedef{PY@tok@cp}{\def\PY@tc##1{\textcolor[rgb]{0.74,0.48,0.00}{##1}}}
\@namedef{PY@tok@k}{\let\PY@bf=\textbf\def\PY@tc##1{\textcolor[rgb]{0.00,0.50,0.00}{##1}}}
\@namedef{PY@tok@kp}{\def\PY@tc##1{\textcolor[rgb]{0.00,0.50,0.00}{##1}}}
\@namedef{PY@tok@kt}{\def\PY@tc##1{\textcolor[rgb]{0.69,0.00,0.25}{##1}}}
\@namedef{PY@tok@o}{\def\PY@tc##1{\textcolor[rgb]{0.40,0.40,0.40}{##1}}}
\@namedef{PY@tok@ow}{\let\PY@bf=\textbf\def\PY@tc##1{\textcolor[rgb]{0.67,0.13,1.00}{##1}}}
\@namedef{PY@tok@nb}{\def\PY@tc##1{\textcolor[rgb]{0.00,0.50,0.00}{##1}}}
\@namedef{PY@tok@nf}{\def\PY@tc##1{\textcolor[rgb]{0.00,0.00,1.00}{##1}}}
\@namedef{PY@tok@nc}{\let\PY@bf=\textbf\def\PY@tc##1{\textcolor[rgb]{0.00,0.00,1.00}{##1}}}
\@namedef{PY@tok@nn}{\let\PY@bf=\textbf\def\PY@tc##1{\textcolor[rgb]{0.00,0.00,1.00}{##1}}}
\@namedef{PY@tok@ne}{\let\PY@bf=\textbf\def\PY@tc##1{\textcolor[rgb]{0.82,0.25,0.23}{##1}}}
\@namedef{PY@tok@nv}{\def\PY@tc##1{\textcolor[rgb]{0.10,0.09,0.49}{##1}}}
\@namedef{PY@tok@no}{\def\PY@tc##1{\textcolor[rgb]{0.53,0.00,0.00}{##1}}}
\@namedef{PY@tok@nl}{\def\PY@tc##1{\textcolor[rgb]{0.63,0.63,0.00}{##1}}}
\@namedef{PY@tok@ni}{\let\PY@bf=\textbf\def\PY@tc##1{\textcolor[rgb]{0.60,0.60,0.60}{##1}}}
\@namedef{PY@tok@na}{\def\PY@tc##1{\textcolor[rgb]{0.49,0.56,0.16}{##1}}}
\@namedef{PY@tok@nt}{\let\PY@bf=\textbf\def\PY@tc##1{\textcolor[rgb]{0.00,0.50,0.00}{##1}}}
\@namedef{PY@tok@nd}{\def\PY@tc##1{\textcolor[rgb]{0.67,0.13,1.00}{##1}}}
\@namedef{PY@tok@s}{\def\PY@tc##1{\textcolor[rgb]{0.73,0.13,0.13}{##1}}}
\@namedef{PY@tok@sd}{\let\PY@it=\textit\def\PY@tc##1{\textcolor[rgb]{0.73,0.13,0.13}{##1}}}
\@namedef{PY@tok@si}{\let\PY@bf=\textbf\def\PY@tc##1{\textcolor[rgb]{0.73,0.40,0.53}{##1}}}
\@namedef{PY@tok@se}{\let\PY@bf=\textbf\def\PY@tc##1{\textcolor[rgb]{0.73,0.40,0.13}{##1}}}
\@namedef{PY@tok@sr}{\def\PY@tc##1{\textcolor[rgb]{0.73,0.40,0.53}{##1}}}
\@namedef{PY@tok@ss}{\def\PY@tc##1{\textcolor[rgb]{0.10,0.09,0.49}{##1}}}
\@namedef{PY@tok@sx}{\def\PY@tc##1{\textcolor[rgb]{0.00,0.50,0.00}{##1}}}
\@namedef{PY@tok@m}{\def\PY@tc##1{\textcolor[rgb]{0.40,0.40,0.40}{##1}}}
\@namedef{PY@tok@gh}{\let\PY@bf=\textbf\def\PY@tc##1{\textcolor[rgb]{0.00,0.00,0.50}{##1}}}
\@namedef{PY@tok@gu}{\let\PY@bf=\textbf\def\PY@tc##1{\textcolor[rgb]{0.50,0.00,0.50}{##1}}}
\@namedef{PY@tok@gd}{\def\PY@tc##1{\textcolor[rgb]{0.63,0.00,0.00}{##1}}}
\@namedef{PY@tok@gi}{\def\PY@tc##1{\textcolor[rgb]{0.00,0.63,0.00}{##1}}}
\@namedef{PY@tok@gr}{\def\PY@tc##1{\textcolor[rgb]{1.00,0.00,0.00}{##1}}}
\@namedef{PY@tok@ge}{\let\PY@it=\textit}
\@namedef{PY@tok@gs}{\let\PY@bf=\textbf}
\@namedef{PY@tok@gp}{\let\PY@bf=\textbf\def\PY@tc##1{\textcolor[rgb]{0.00,0.00,0.50}{##1}}}
\@namedef{PY@tok@go}{\def\PY@tc##1{\textcolor[rgb]{0.53,0.53,0.53}{##1}}}
\@namedef{PY@tok@gt}{\def\PY@tc##1{\textcolor[rgb]{0.00,0.27,0.87}{##1}}}
\@namedef{PY@tok@err}{\def\PY@bc##1{{\setlength{\fboxsep}{\string -\fboxrule}\fcolorbox[rgb]{1.00,0.00,0.00}{1,1,1}{\strut ##1}}}}
\@namedef{PY@tok@kc}{\let\PY@bf=\textbf\def\PY@tc##1{\textcolor[rgb]{0.00,0.50,0.00}{##1}}}
\@namedef{PY@tok@kd}{\let\PY@bf=\textbf\def\PY@tc##1{\textcolor[rgb]{0.00,0.50,0.00}{##1}}}
\@namedef{PY@tok@kn}{\let\PY@bf=\textbf\def\PY@tc##1{\textcolor[rgb]{0.00,0.50,0.00}{##1}}}
\@namedef{PY@tok@kr}{\let\PY@bf=\textbf\def\PY@tc##1{\textcolor[rgb]{0.00,0.50,0.00}{##1}}}
\@namedef{PY@tok@bp}{\def\PY@tc##1{\textcolor[rgb]{0.00,0.50,0.00}{##1}}}
\@namedef{PY@tok@fm}{\def\PY@tc##1{\textcolor[rgb]{0.00,0.00,1.00}{##1}}}
\@namedef{PY@tok@vc}{\def\PY@tc##1{\textcolor[rgb]{0.10,0.09,0.49}{##1}}}
\@namedef{PY@tok@vg}{\def\PY@tc##1{\textcolor[rgb]{0.10,0.09,0.49}{##1}}}
\@namedef{PY@tok@vi}{\def\PY@tc##1{\textcolor[rgb]{0.10,0.09,0.49}{##1}}}
\@namedef{PY@tok@vm}{\def\PY@tc##1{\textcolor[rgb]{0.10,0.09,0.49}{##1}}}
\@namedef{PY@tok@sa}{\def\PY@tc##1{\textcolor[rgb]{0.73,0.13,0.13}{##1}}}
\@namedef{PY@tok@sb}{\def\PY@tc##1{\textcolor[rgb]{0.73,0.13,0.13}{##1}}}
\@namedef{PY@tok@sc}{\def\PY@tc##1{\textcolor[rgb]{0.73,0.13,0.13}{##1}}}
\@namedef{PY@tok@dl}{\def\PY@tc##1{\textcolor[rgb]{0.73,0.13,0.13}{##1}}}
\@namedef{PY@tok@s2}{\def\PY@tc##1{\textcolor[rgb]{0.73,0.13,0.13}{##1}}}
\@namedef{PY@tok@sh}{\def\PY@tc##1{\textcolor[rgb]{0.73,0.13,0.13}{##1}}}
\@namedef{PY@tok@s1}{\def\PY@tc##1{\textcolor[rgb]{0.73,0.13,0.13}{##1}}}
\@namedef{PY@tok@mb}{\def\PY@tc##1{\textcolor[rgb]{0.40,0.40,0.40}{##1}}}
\@namedef{PY@tok@mf}{\def\PY@tc##1{\textcolor[rgb]{0.40,0.40,0.40}{##1}}}
\@namedef{PY@tok@mh}{\def\PY@tc##1{\textcolor[rgb]{0.40,0.40,0.40}{##1}}}
\@namedef{PY@tok@mi}{\def\PY@tc##1{\textcolor[rgb]{0.40,0.40,0.40}{##1}}}
\@namedef{PY@tok@il}{\def\PY@tc##1{\textcolor[rgb]{0.40,0.40,0.40}{##1}}}
\@namedef{PY@tok@mo}{\def\PY@tc##1{\textcolor[rgb]{0.40,0.40,0.40}{##1}}}
\@namedef{PY@tok@ch}{\let\PY@it=\textit\def\PY@tc##1{\textcolor[rgb]{0.25,0.50,0.50}{##1}}}
\@namedef{PY@tok@cm}{\let\PY@it=\textit\def\PY@tc##1{\textcolor[rgb]{0.25,0.50,0.50}{##1}}}
\@namedef{PY@tok@cpf}{\let\PY@it=\textit\def\PY@tc##1{\textcolor[rgb]{0.25,0.50,0.50}{##1}}}
\@namedef{PY@tok@c1}{\let\PY@it=\textit\def\PY@tc##1{\textcolor[rgb]{0.25,0.50,0.50}{##1}}}
\@namedef{PY@tok@cs}{\let\PY@it=\textit\def\PY@tc##1{\textcolor[rgb]{0.25,0.50,0.50}{##1}}}

\def\PYZbs{\char`\\}
\def\PYZus{\char`\_}
\def\PYZob{\char`\{}
\def\PYZcb{\char`\}}
\def\PYZca{\char`\^}
\def\PYZam{\char`\&}
\def\PYZlt{\char`\<}
\def\PYZgt{\char`\>}
\def\PYZsh{\char`\#}
\def\PYZpc{\char`\%}
\def\PYZdl{\char`\$}
\def\PYZhy{\char`\-}
\def\PYZsq{\char`\'}
\def\PYZdq{\char`\"}
\def\PYZti{\char`\~}
% for compatibility with earlier versions
\def\PYZat{@}
\def\PYZlb{[}
\def\PYZrb{]}
\makeatother


    % For linebreaks inside Verbatim environment from package fancyvrb. 
    \makeatletter
        \newbox\Wrappedcontinuationbox 
        \newbox\Wrappedvisiblespacebox 
        \newcommand*\Wrappedvisiblespace {\textcolor{red}{\textvisiblespace}} 
        \newcommand*\Wrappedcontinuationsymbol {\textcolor{red}{\llap{\tiny$\m@th\hookrightarrow$}}} 
        \newcommand*\Wrappedcontinuationindent {3ex } 
        \newcommand*\Wrappedafterbreak {\kern\Wrappedcontinuationindent\copy\Wrappedcontinuationbox} 
        % Take advantage of the already applied Pygments mark-up to insert 
        % potential linebreaks for TeX processing. 
        %        {, <, #, %, $, ' and ": go to next line. 
        %        _, }, ^, &, >, - and ~: stay at end of broken line. 
        % Use of \textquotesingle for straight quote. 
        \newcommand*\Wrappedbreaksatspecials {% 
            \def\PYGZus{\discretionary{\char`\_}{\Wrappedafterbreak}{\char`\_}}% 
            \def\PYGZob{\discretionary{}{\Wrappedafterbreak\char`\{}{\char`\{}}% 
            \def\PYGZcb{\discretionary{\char`\}}{\Wrappedafterbreak}{\char`\}}}% 
            \def\PYGZca{\discretionary{\char`\^}{\Wrappedafterbreak}{\char`\^}}% 
            \def\PYGZam{\discretionary{\char`\&}{\Wrappedafterbreak}{\char`\&}}% 
            \def\PYGZlt{\discretionary{}{\Wrappedafterbreak\char`\<}{\char`\<}}% 
            \def\PYGZgt{\discretionary{\char`\>}{\Wrappedafterbreak}{\char`\>}}% 
            \def\PYGZsh{\discretionary{}{\Wrappedafterbreak\char`\#}{\char`\#}}% 
            \def\PYGZpc{\discretionary{}{\Wrappedafterbreak\char`\%}{\char`\%}}% 
            \def\PYGZdl{\discretionary{}{\Wrappedafterbreak\char`\$}{\char`\$}}% 
            \def\PYGZhy{\discretionary{\char`\-}{\Wrappedafterbreak}{\char`\-}}% 
            \def\PYGZsq{\discretionary{}{\Wrappedafterbreak\textquotesingle}{\textquotesingle}}% 
            \def\PYGZdq{\discretionary{}{\Wrappedafterbreak\char`\"}{\char`\"}}% 
            \def\PYGZti{\discretionary{\char`\~}{\Wrappedafterbreak}{\char`\~}}% 
        } 
        % Some characters . , ; ? ! / are not pygmentized. 
        % This macro makes them "active" and they will insert potential linebreaks 
        \newcommand*\Wrappedbreaksatpunct {% 
            \lccode`\~`\.\lowercase{\def~}{\discretionary{\hbox{\char`\.}}{\Wrappedafterbreak}{\hbox{\char`\.}}}% 
            \lccode`\~`\,\lowercase{\def~}{\discretionary{\hbox{\char`\,}}{\Wrappedafterbreak}{\hbox{\char`\,}}}% 
            \lccode`\~`\;\lowercase{\def~}{\discretionary{\hbox{\char`\;}}{\Wrappedafterbreak}{\hbox{\char`\;}}}% 
            \lccode`\~`\:\lowercase{\def~}{\discretionary{\hbox{\char`\:}}{\Wrappedafterbreak}{\hbox{\char`\:}}}% 
            \lccode`\~`\?\lowercase{\def~}{\discretionary{\hbox{\char`\?}}{\Wrappedafterbreak}{\hbox{\char`\?}}}% 
            \lccode`\~`\!\lowercase{\def~}{\discretionary{\hbox{\char`\!}}{\Wrappedafterbreak}{\hbox{\char`\!}}}% 
            \lccode`\~`\/\lowercase{\def~}{\discretionary{\hbox{\char`\/}}{\Wrappedafterbreak}{\hbox{\char`\/}}}% 
            \catcode`\.\active
            \catcode`\,\active 
            \catcode`\;\active
            \catcode`\:\active
            \catcode`\?\active
            \catcode`\!\active
            \catcode`\/\active 
            \lccode`\~`\~ 	
        }
    \makeatother

    \let\OriginalVerbatim=\Verbatim
    \makeatletter
    \renewcommand{\Verbatim}[1][1]{%
        %\parskip\z@skip
        \sbox\Wrappedcontinuationbox {\Wrappedcontinuationsymbol}%
        \sbox\Wrappedvisiblespacebox {\FV@SetupFont\Wrappedvisiblespace}%
        \def\FancyVerbFormatLine ##1{\hsize\linewidth
            \vtop{\raggedright\hyphenpenalty\z@\exhyphenpenalty\z@
                \doublehyphendemerits\z@\finalhyphendemerits\z@
                \strut ##1\strut}%
        }%
        % If the linebreak is at a space, the latter will be displayed as visible
        % space at end of first line, and a continuation symbol starts next line.
        % Stretch/shrink are however usually zero for typewriter font.
        \def\FV@Space {%
            \nobreak\hskip\z@ plus\fontdimen3\font minus\fontdimen4\font
            \discretionary{\copy\Wrappedvisiblespacebox}{\Wrappedafterbreak}
            {\kern\fontdimen2\font}%
        }%
        
        % Allow breaks at special characters using \PYG... macros.
        \Wrappedbreaksatspecials
        % Breaks at punctuation characters . , ; ? ! and / need catcode=\active 	
        \OriginalVerbatim[#1,codes*=\Wrappedbreaksatpunct]%
    }
    \makeatother

    % Exact colors from NB
    \definecolor{incolor}{HTML}{303F9F}
    \definecolor{outcolor}{HTML}{D84315}
    \definecolor{cellborder}{HTML}{CFCFCF}
    \definecolor{cellbackground}{HTML}{F7F7F7}
    
    % prompt
    \makeatletter
    \newcommand{\boxspacing}{\kern\kvtcb@left@rule\kern\kvtcb@boxsep}
    \makeatother
    \newcommand{\prompt}[4]{
        {\ttfamily\llap{{\color{#2}[#3]:\hspace{3pt}#4}}\vspace{-\baselineskip}}
    }
    

    
    % Prevent overflowing lines due to hard-to-break entities
    \sloppy 
    % Setup hyperref package
    \hypersetup{
      breaklinks=true,  % so long urls are correctly broken across lines
      colorlinks=true,
      urlcolor=urlcolor,
      linkcolor=linkcolor,
      citecolor=citecolor,
      }
    % Slightly bigger margins than the latex defaults
    
    \geometry{verbose,tmargin=1in,bmargin=1in,lmargin=1in,rmargin=1in}
    
    
\date{}
% \setcounter{page}{2}

\begin{document}
\setcounter{page}{2}

\hypertarget{objectives}{%
  \section{Objectives}\label{objectives}}

\begin{enumerate}
  \def\labelenumi{\arabic{enumi}.}
  \item
        To determine heat of mixing of binary mixture using an adiabatic
        caloriemeter
  \item
        To determine the water equivalent of calorimeter and finding the
        specific heat capacity of solution.
  \item
        To determine the partial molar enthalpy of water and acetone at x=0.5
        and partial enthalpy of mixing at infinite dilution.
\end{enumerate}

\hypertarget{notes-on-the-procedure}{%
  \section{Notes on the Procedure}\label{notes-on-the-procedure}}

\begin{enumerate}
  \def\labelenumi{\arabic{enumi}.}
  \tightlist
  \item
        Rinse the cylindrical container using the chemical to be poured in it.
        Reserve one container for each chemical.
  \item
        Check whether the stirrer works properly.
  \item
        Calculate the volume of each component required for each value of
        concentration.
  \item
        Determine the water equivalent of the flask by taking 150ml of
        distilled water
  \item
        Pour the required volume of bulk solvent (component with higher mole
        fraction) into the adiabatic calorimeter and take the steady
        temperature reading (T0).~
  \item
        Pour the required volume of other component in the adiabatic
        calorimeter and start the power source and stopwatch. Note the
        voltage, current and temperature reading at this starting point (T1).
  \item
        Note the time taken for 2\(^{\circ}\)C rise in temperature by stopping
        stopwatch after 2\(^{\circ}\)C rise.
  \item
        Dispose off the solution in the waste solution container.
  \item
        Wipe off the heating element and the temperature sensor properly
        before taking the next reading
  \item
        Repeat the experiment for different set of values of mole fraction.~
\end{enumerate}

\hypertarget{set-up}{%
  \section{Set-up}\label{set-up}}

\begin{enumerate}
  \def\labelenumi{\arabic{enumi}.}
  \tightlist
  \item
        Adiabatic Flask
  \item
        Heating element having a heating coil and a temperature sensor
  \item
        Temperature Display connected to the temperature sensor
  \item
        Magnetic stirrer and bead for homogenisation of the solution
  \item
        Timer - To keep track of time
  \item
        Voltage and Current regulator
  \item
        Acetone
  \item
        Measuring flask
\end{enumerate}

\pagebreak
\hypertarget{results}{%
  \section{Results}\label{results}}

\hypertarget{raw-data}{%
  \subsection{Raw Data}\label{raw-data}}

\textbf{Table 1: Values of quantities used from literature}

\begin{table}[h]
  \centering
  \begin{tabular}{|c|c|}
    \hline
    Quantity                                  & Value              \\
    \hline
    Density of water                          & 997\(kg/m^3\)      \\
    \hline
    Density of acetone                        & 790\(kg/m^3\)      \\
    \hline
    Heat capacity of water at 20\(^{\circ}\)C & 1 cal/gK=4.186J/gK \\
    \hline
  \end{tabular}
\end{table}

\textbf{Table 2: Volume of water and acetone required for making
  solutions of corresponding concentration}

\begin{table}[h]
  \centering
  \begin{tabular}{|c|c|c|}
    \hline
    Mole fraction of acetone & Volume of water (\(V_w\)) & Volume of acetone (\(V_a\)) \\
    \hline
    0                        & 150                       & 0                           \\
    \hline
    0.1                      & 102.94                    & 47.06                       \\
    \hline
    0.3                      & 54.28                     & 95.72                       \\
    \hline
    0.5                      & 29.4                      & 120.6                       \\
    \hline
    0.7                      & 14.15                     & 135.85                      \\
    \hline
    0.9                      & 3.95                      & 146.05                      \\
    \hline
  \end{tabular}
\end{table}

\textbf{Table 3: Measured quantities}

\begin{center}
  \adjustimage{max size={0.95\linewidth}{0.95\paperheight}}{table3.jpeg}
\end{center}

\hypertarget{calculations}{%
  \subsection{Calculations}\label{calculations}}

\hypertarget{determining-water-equivalent-k-of-the-calorimeter}{%
  \subsubsection{Determining water equivalent (K) of the
    calorimeter}\label{determining-water-equivalent-k-of-the-calorimeter}}

The relevant equation is:

\begin{equation}
  \begin{split}
    Q(input) &= m_wC_p\Delta T+K\Delta T\\
    IVt&=\rho_wV_wC_p\Delta T+K\Delta T
  \end{split}
\end{equation}

where w represents `water'. Substituting vales, \(K=1275.22g=1.275\)kg.

\hypertarget{volume-of-water-and-acetone-required-for-sample-1}{%
  \subsubsection{Volume of water and acetone required for Sample
    1}\label{volume-of-water-and-acetone-required-for-sample-1}}

Let \(n_1\)\& \(n_2\) represent the number of moles of acetone and water
respectively. For the first sample:

\begin{equation}
  \frac{n_1}{n_1+n_2}=x(=0.1/0.3/0.5/0.7/0.9)
  \implies (1-x)n_1=xn_2
\end{equation}

If m and M represent mass and molar mass respectively:

\begin{equation}
  \begin{split}
    \frac{(1-x)m_1}{M_1}&=\frac{xm_2}{M_2}\\
    \frac{(1-x)\rho_1V_1}{M_1}&=\frac{x\rho_2V_2}{M_2}
  \end{split}
\end{equation}

Further, since \(V_1+V_2=150mL\), we have:

\begin{equation}
  \begin{split}
    \frac{(1-x)\rho_1V_1}{M_1}&=\frac{x\rho_2(150-V_1)}{M_2}
  \end{split}
\end{equation}

Plugging in the values (\(x=0.1\) for first sample) led to
\(V_1=47.06\)mL. Similarly for other samples.

\hypertarget{mixing-enthalpy-for-sample-1}{%
  \subsubsection{Mixing enthalpy for Sample
    1}\label{mixing-enthalpy-for-sample-1}}

The relevant equation is:

\begin{equation}
  Q = (mCp+K)\Delta T\implies mCp+K=Q/2=IVt/2
\end{equation}

Next, we also have:

\begin{equation}
  \begin{split}
    Q_m=-(mCp+K)\Delta T_m&=-IVt\Delta T_m/2\\
    \therefore \Delta H_m=\frac{-IVt\Delta T_m}{2(n_1+n_2)}
  \end{split}
\end{equation}

For sample 1, the value obtained was \(\Delta H_m=-1923.98\)J/mol

\hypertarget{partial-molar-enthalpies-at-x_1x_20.5}{%
  \subsubsection{\texorpdfstring{Partial molar enthalpies at
      \(x_1=x_2=0.5\)}{Partial molar enthalpies at x\_1=x\_2=0.5}}\label{partial-molar-enthalpies-at-x_1x_20.5}}

The required expressions:

\begin{equation}
  \begin{split}
    H_{1p}&=\Delta H_m+x_2\frac{d\Delta H_m}{dx_1}\\
    H_{2p}&=\Delta H_m-x_1\frac{d\Delta H_m}{dx_1}
  \end{split}
\end{equation}

At \(x_1=x_2=0.5, \frac{d\Delta H_m}{dx_1}=2294.17\)J/mol and thus:

\begin{equation}
  \begin{split}
    H_{1p}&=-1088.21+(0.5)(2294.17)=58.875J/mol\\
    H_{2p}&=-1088.21-(0.5)(2294.17)=-2235.29J/mol
  \end{split}
\end{equation}

\hypertarget{infinite-dilution-partial-molar-enthalpies}{%
  \subsubsection{Infinite dilution partial molar
    enthalpies}\label{infinite-dilution-partial-molar-enthalpies}}

By letting the mole fractions tend to limits 1 and 0:

\begin{equation}
  \begin{split}
    H_{1\infty}(x_1\to0)&=\Delta H_m+(1)\frac{d\Delta H_m}{dx_1}=-31521.02J/mol\\
    H_{2\infty}(x_1\to1)&=\Delta H_m-(1)\frac{d\Delta H_m}{dx_1}=-4913.55J/mol
  \end{split}
\end{equation}

\textbf{Table 4: Derived Quantities}

\begin{center}
  \adjustimage{max size={0.9\linewidth}{0.9\paperheight}}{table4.jpg}
\end{center}

\hypertarget{error-analysis}{%
  \subsubsection{Error Analysis}\label{error-analysis}}

\textbf{Table 5: Least Counts}

\begin{table}[h]
  \centering
  \begin{tabular}{|c|c|}
    \hline
    Instrument         & Least Count      \\
    \hline
    Measuring Cylinder & 1mL              \\
    \hline
    Pipette            & 0.02mL           \\
    \hline
    Stopwatch          & 0.167s           \\
    \hline
    Voltmeter          & 0.01V            \\
    \hline
    Ammeter            & 0.01A            \\
    \hline
    Temperature Sensor & 0.1\(^{\circ}C\) \\
    \hline
    Weighing Scale     & 0.1g             \\
    \hline
  \end{tabular}
\end{table}

The error propagation is:

\begin{equation}
  \begin{split}
    \frac{\Delta k}{k}&=\frac{\Delta I}{I}+\frac{\Delta V}{V}+\frac{\Delta t}{t}+\frac{\Delta (\Delta T)}{\Delta T}+\frac{\Delta m}{m}\\
    \frac{\Delta Cp}{Cp}&=\frac{\Delta k}{k}+\frac{\Delta I}{I}+\frac{\Delta V}{V}+\frac{\Delta t}{t}+\frac{\Delta (\Delta T)}{\Delta T}+\frac{\Delta m}{m} \\
    \frac{\Delta (\Delta H_m)}{\Delta H_m}&=\frac{\Delta k}{k}+\frac{\Delta Cp}{Cp}+\frac{\Delta(\Delta T)}{\Delta T}+\frac{\Delta m}{m}
  \end{split}
\end{equation}

Plugging in values and least count led to an error of 8.41\% for the
first sample.

\hypertarget{discussion}{%
  \section{Discussion}\label{discussion}}

\hypertarget{observations}{%
  \subsection{Observations}\label{observations}}

\begin{enumerate}
  \def\labelenumi{\arabic{enumi}.}
  \tightlist
  \item
        Time taken to obtain 2\(^{\circ}\)C rise in temperature of solution
        decreases with increase in mole fraction of acetone
  \item
        As concentration of acetone increases the enthalpy of mixing becomes
        less negative
\end{enumerate}

\hypertarget{questions}{%
  \subsection{Questions}\label{questions}}

Perform the experiment to test the following hypotheses:

\begin{enumerate}
  \def\labelenumi{\arabic{enumi}.}
  \tightlist
  \item
        The calorimeter functions adiabatically
  \item
        A temperature increase of 2\(^{\circ}\)C is sufficient to estimate
        ``K'' with \textless{} 5\% uncertainty (Uncertainty decreases if
        dynamic std-dev is used instead of least count)
  \item
        For a given 2-component system mixing can only be an exothermic or
        endothermic process
  \item
        Find the heat capacity of the pure component along with its
        uncertainty. Compare with literature value and comment
  \item
        Find the partial molar enthalpy of mixing at \(x_1=x_2=0.5\) and at
        infinite dilution.
\end{enumerate}

\hypertarget{hypotheses}{%
  \subsection{Hypotheses}\label{hypotheses}}

\hypertarget{error-in-estimation-of-k}{%
  \subsubsection{Error in estimation of
    K}\label{error-in-estimation-of-k}}

Based on the data provided for calculating water equivalent and errors,
the error in estimation was calculated as follows:

\begin{equation}
  \begin{split}
    \frac{\Delta k}{k}&=\frac{\Delta I}{I}+\frac{\Delta V}{V}+\frac{\Delta t}{t}+\frac{\Delta (\Delta T)}{\Delta T}+\frac{\Delta m}{m}\\
    \frac{\Delta k}{k}&=\left[\frac{0.01}{2.7}+\frac{0.01}{16.3}+\frac{0.167}{86.4}+\frac{0.1}{2}+\frac{0.1}{150\times0.997}\right]\times100\%=5.69\%\\
  \end{split}
\end{equation}

Thus, the hypothesis that a temperature rise of 2\(^{\circ}\)C is enough
to estimate K with \(<5\%\) accuracy is \textbf{incorrect}.

However measurement includes a standard uncertainty which is approximately
1 standard deviation(ie least count) which led to error of 5.7\%. If however,
we use dynamic standard deviations (say $3\sigma$) then almost 99.7\% of
the readings are covered(ie uncertainty of 0.3\%), ie the error in K can
be easily estimated to be $\leq5\%$.

\hypertarget{heat-of-mixing}{%
  \subsubsection{Heat of mixing}\label{heat-of-mixing}}

Heat of mixing (\(\Delta H_m\)) is defined as the change in enthalpy
observed when two liquids are mixed, being the sum of the changes in
enthalpy which occur during the mixing process. It primarily depends
upon the interactions of the molecules of the substances involved and
the ratio they are mixed in. Mixing enthalpy is zero if the mixture is
ideal (if there are no interactions between the molecules). However,
interactions between the two components can cause endothermic effects or
exothermic effects depending on the relative attractions of molecules.
In this case (water-acetone mixture), hydrogen bonding increases with
increase in the concentration of acetone up to a certain
concentration/mole fraction of acetone. However, as acetone molecules
come to the surface, the hydrophobic acetone group forms clusters and
hence reduces its contact with water. Till the point where no clusters
are formed the hydrogen bonding will lead to release of energy hence
making the process exothermic. But after a particlular concentration the
clusters will lead to hydropibicity hence the mixing becomes
endothermic. (\emph{Although not visible in this data, trends suggest
  such a thing will happen})

\hypertarget{heat-capacity-of-acetone}{%
  \subsubsection{Heat Capacity of
    Acetone}\label{heat-capacity-of-acetone}}

The heat capacity of a substance is defined as the quantity of heat
required to raise its temperature by 1\(^{\circ}\)C (equivalent to 1 K).
It is observed that with increase in mole fraction of acetone, the time
taken to achieve a set temperature rise of 2\(^{\circ}\)C decreases -
this indicates that the heat capacity of the mixture decreases with the
increase in mole fraction of acetone. Heat capacity of the mixture can
be considered a weighted average of heat capacities of the components
(with the number of moles being the respective weights for total heat
capacity and mole fractions being the weights for the molar heat
capacity) and heat capacity decreases with increase in the weightage for
acetone. Therefore, the heat capacity of acetone is less than the heat
capacity of water. This is in fact true, since literature suggests heat
capacity of acetone=\(1.29\) J/g\(.\)K which is definitely less than
water.

\emph{We have not calculated the specific heat of acetone since the data
  provided did not include values for 1M solution.}

\hypertarget{adiabatic-caloriemeter}{%
  \subsubsection{Adiabatic Caloriemeter}\label{adiabatic-caloriemeter}}

Since we have taken one observation for the water equivalence
calculation it is difficult to conclude if the caloriemeter is adiabatic
or not. It can be concluded that if any other parameter (m, Cp, deltaT)
is changed then water equivalent should remain the same.

\hypertarget{code-python}{%
  \section{Code (Python)}\label{code-python}}

\begin{tcolorbox}[breakable, size=fbox, boxrule=1pt, pad at break*=1mm,colback=cellbackground, colframe=cellborder]
  \prompt{In}{incolor}{1}{\boxspacing}
  \begin{Verbatim}[commandchars=\\\{\}]
    \PY{k+kn}{import} \PY{n+nn}{numpy} \PY{k}{as} \PY{n+nn}{np}
    \PY{k+kn}{import} \PY{n+nn}{matplotlib}\PY{n+nn}{.}\PY{n+nn}{pyplot} \PY{k}{as} \PY{n+nn}{plt}
    \PY{k+kn}{from} \PY{n+nn}{scipy} \PY{k+kn}{import} \PY{n}{interpolate}
  \end{Verbatim}
\end{tcolorbox}

\begin{tcolorbox}[breakable, size=fbox, boxrule=1pt, pad at break*=1mm,colback=cellbackground, colframe=cellborder]
  \prompt{In}{incolor}{2}{\boxspacing}
  \begin{Verbatim}[commandchars=\\\{\}]
    \PY{n}{x\PYZus{}a} \PY{o}{=} \PY{n}{np}\PY{o}{.}\PY{n}{array}\PY{p}{(}\PY{p}{[}\PY{l+m+mf}{0.1}\PY{p}{,} \PY{l+m+mf}{0.3}\PY{p}{,} \PY{l+m+mf}{0.5}\PY{p}{,} \PY{l+m+mf}{0.7}\PY{p}{,} \PY{l+m+mf}{0.9}\PY{p}{]}\PY{p}{)}  \PY{c+c1}{\PYZsh{} n\PYZus{}acetone / n\PYZus{}total}
    \PY{n}{V\PYZus{}a} \PY{o}{=} \PY{n}{np}\PY{o}{.}\PY{n}{array}\PY{p}{(}\PY{p}{[}\PY{l+m+mf}{47.06}\PY{p}{,} \PY{l+m+mf}{95.72}\PY{p}{,} \PY{l+m+mf}{120.6}\PY{p}{,} \PY{l+m+mf}{135.85}\PY{p}{,} \PY{l+m+mf}{146.05}\PY{p}{]}\PY{p}{)}
    \PY{n}{m} \PY{o}{=} \PY{n}{np}\PY{o}{.}\PY{n}{array}\PY{p}{(}\PY{p}{[}\PY{l+m+mf}{139.9292}\PY{p}{,} \PY{l+m+mf}{129.5159}\PY{p}{,} \PY{l+m+mf}{124.1916}\PY{p}{,} \PY{l+m+mf}{120.9281}\PY{p}{,} \PY{l+m+mf}{118.7453}\PY{p}{]}\PY{p}{)}
    \PY{n}{n} \PY{o}{=} \PY{n}{np}\PY{o}{.}\PY{n}{array}\PY{p}{(}\PY{p}{[}\PY{l+m+mf}{6.355755}\PY{p}{,} \PY{l+m+mf}{4.31094}\PY{p}{,} \PY{l+m+mf}{3.26542}\PY{p}{,} \PY{l+m+mf}{2.624577}\PY{p}{,} \PY{l+m+mf}{2.195948}\PY{p}{]}\PY{p}{)}
    \PY{n}{T\PYZus{}i} \PY{o}{=} \PY{n}{np}\PY{o}{.}\PY{n}{array}\PY{p}{(}\PY{p}{[}\PY{l+m+mf}{29.2}\PY{p}{,} \PY{l+m+mf}{29.3}\PY{p}{,} \PY{l+m+mf}{29.2}\PY{p}{,} \PY{l+m+mf}{29.3}\PY{p}{,} \PY{l+m+mf}{29.3}\PY{p}{]}\PY{p}{)}
    \PY{n}{T\PYZus{}mi} \PY{o}{=} \PY{n}{np}\PY{o}{.}\PY{n}{array}\PY{p}{(}\PY{p}{[}\PY{l+m+mf}{35.6}\PY{p}{,} \PY{l+m+mf}{33.4}\PY{p}{,} \PY{l+m+mf}{31.7}\PY{p}{,} \PY{l+m+mf}{30.3}\PY{p}{,} \PY{l+m+mf}{29.9}\PY{p}{]}\PY{p}{)}
    \PY{n}{dT\PYZus{}m} \PY{o}{=} \PY{l+m+mi}{2}

    \PY{n}{delta\PYZus{}T\PYZus{}m} \PY{o}{=} \PY{n}{np}\PY{o}{.}\PY{n}{array}\PY{p}{(}\PY{p}{[}\PY{l+m+mf}{6.4}\PY{p}{,} \PY{l+m+mf}{4.1}\PY{p}{,} \PY{l+m+mf}{2.5}\PY{p}{,} \PY{l+m+mi}{1}\PY{p}{,} \PY{l+m+mf}{0.6}\PY{p}{]}\PY{p}{)}

    \PY{n}{V} \PY{o}{=} \PY{l+m+mf}{16.4} \PY{c+c1}{\PYZsh{} volts}
    \PY{n}{I} \PY{o}{=} \PY{l+m+mf}{2.7} \PY{c+c1}{\PYZsh{} ampere}

    \PY{n}{t} \PY{o}{=} \PY{n}{np}\PY{o}{.}\PY{n}{array}\PY{p}{(}\PY{p}{[}\PY{l+m+mf}{86.3}\PY{p}{,} \PY{l+m+mf}{72.89}\PY{p}{,} \PY{l+m+mf}{64.2}\PY{p}{,} \PY{l+m+mf}{56.73}\PY{p}{,} \PY{l+m+mf}{49.8}\PY{p}{]}\PY{p}{)}
  \end{Verbatim}
\end{tcolorbox}

\begin{tcolorbox}[breakable, size=fbox, boxrule=1pt, pad at break*=1mm,colback=cellbackground, colframe=cellborder]
  \prompt{In}{incolor}{3}{\boxspacing}
  \begin{Verbatim}[commandchars=\\\{\}]
    \PY{n}{V\PYZus{}tot} \PY{o}{=} \PY{l+m+mi}{150}

    \PY{n}{M\PYZus{}a} \PY{o}{=} \PY{l+m+mi}{58}
    \PY{n}{rho\PYZus{}a} \PY{o}{=} \PY{l+m+mi}{784} \PY{c+c1}{\PYZsh{} kg / m³}

    \PY{n}{M\PYZus{}w} \PY{o}{=} \PY{l+m+mi}{18}
    \PY{n}{rho\PYZus{}w} \PY{o}{=} \PY{l+m+mi}{997} \PY{c+c1}{\PYZsh{} kg / m³}
  \end{Verbatim}
\end{tcolorbox}

\begin{tcolorbox}[breakable, size=fbox, boxrule=1pt, pad at break*=1mm,colback=cellbackground, colframe=cellborder]
  \prompt{In}{incolor}{4}{\boxspacing}
  \begin{Verbatim}[commandchars=\\\{\}]
    \PY{n}{dH\PYZus{}m} \PY{o}{=} \PY{o}{\PYZhy{}} \PY{n}{I} \PY{o}{*} \PY{n}{V} \PY{o}{*} \PY{n}{t} \PY{o}{*} \PY{n}{delta\PYZus{}T\PYZus{}m} \PY{o}{/} \PY{l+m+mi}{2}
    \PY{n}{dH\PYZus{}m\PYZus{}permole} \PY{o}{=} \PY{n}{dH\PYZus{}m} \PY{o}{/} \PY{n}{n}
    \PY{n}{dH\PYZus{}m\PYZus{}permole}
  \end{Verbatim}
\end{tcolorbox}

\begin{tcolorbox}[breakable, size=fbox, boxrule=.5pt, pad at break*=1mm, opacityfill=0]
  \prompt{Out}{outcolor}{4}{\boxspacing}
  \begin{Verbatim}[commandchars=\\\{\}]
    array([-1923.98303585, -1534.81998358, -1088.2122361 ,  -478.55414415,
        -301.2563139 ])
  \end{Verbatim}
\end{tcolorbox}

\begin{tcolorbox}[breakable, size=fbox, boxrule=1pt, pad at break*=1mm,colback=cellbackground, colframe=cellborder]
  \prompt{In}{incolor}{5}{\boxspacing}
  \begin{Verbatim}[commandchars=\\\{\}]
    \PY{n}{LC\PYZus{}I} \PY{o}{=} \PY{l+m+mf}{0.01} \PY{c+c1}{\PYZsh{} A}
    \PY{n}{LC\PYZus{}V} \PY{o}{=} \PY{l+m+mf}{0.01} \PY{c+c1}{\PYZsh{} V}
    \PY{n}{LC\PYZus{}t} \PY{o}{=} \PY{l+m+mf}{0.167} \PY{c+c1}{\PYZsh{} s}
    \PY{n}{LC\PYZus{}m} \PY{o}{=} \PY{l+m+mf}{0.1} \PY{c+c1}{\PYZsh{} g}
    \PY{n}{LC\PYZus{}DT} \PY{o}{=} \PY{l+m+mf}{0.1} \PY{c+c1}{\PYZsh{} deg C}
    \PY{n}{LC\PYZus{}V\PYZus{}a} \PY{o}{=} \PY{l+m+mi}{2}

    \PY{c+c1}{\PYZsh{} Testing hypothesis}
    \PY{c+c1}{\PYZsh{} Delta\PYZus{}DT = LC\PYZus{}DT / delta\PYZus{}T\PYZus{}m}
    \PY{n}{Delta\PYZus{}DT} \PY{o}{=} \PY{l+m+mf}{0.1} \PY{o}{/} \PY{l+m+mi}{2} \PY{c+c1}{\PYZsh{} 5 \PYZpc{} error}

    \PY{n}{LC\PYZus{}k\PYZus{}by\PYZus{}k} \PY{o}{=} \PY{n}{LC\PYZus{}I} \PY{o}{/} \PY{n}{I} \PY{o}{+} \PY{n}{LC\PYZus{}V} \PY{o}{/} \PY{n}{V}  \PY{o}{+} \PY{n}{LC\PYZus{}t} \PY{o}{/} \PY{n}{t} \PY{o}{+} \PY{n}{Delta\PYZus{}DT} \PY{o}{+} \PY{n}{LC\PYZus{}m} \PY{o}{/} \PY{n}{m}
    \PY{n}{LC\PYZus{}Cp\PYZus{}by\PYZus{}Cp} \PY{o}{=} \PY{n}{LC\PYZus{}k\PYZus{}by\PYZus{}k} \PY{o}{+} \PY{n}{LC\PYZus{}I} \PY{o}{/} \PY{n}{I} \PY{o}{+} \PY{n}{LC\PYZus{}V} \PY{o}{/} \PY{n}{V}  \PY{o}{+} \PY{n}{LC\PYZus{}t} \PY{o}{/} \PY{n}{t} \PY{o}{+} \PY{n}{LC\PYZus{}DT} \PY{o}{/} \PY{n}{delta\PYZus{}T\PYZus{}m} \PY{o}{+} \PY{n}{LC\PYZus{}m} \PY{o}{/} \PY{n}{m}

    \PY{n}{LC\PYZus{}error} \PY{o}{=} \PY{n}{LC\PYZus{}k\PYZus{}by\PYZus{}k} \PY{o}{+} \PY{n}{LC\PYZus{}Cp\PYZus{}by\PYZus{}Cp} \PY{o}{+} \PY{n}{LC\PYZus{}DT} \PY{o}{/} \PY{n}{delta\PYZus{}T\PYZus{}m} \PY{o}{+} \PY{n}{LC\PYZus{}m} \PY{o}{/} \PY{n}{m}
    \PY{n+nb}{print}\PY{p}{(}\PY{l+s+s1}{\PYZsq{}}\PY{l+s+s1}{Total error: }\PY{l+s+s1}{\PYZsq{}}\PY{p}{,} \PY{n}{LC\PYZus{}error} \PY{o}{*} \PY{l+m+mi}{100}\PY{p}{)}
    \PY{n+nb}{print}\PY{p}{(}\PY{l+s+s1}{\PYZsq{}}\PY{l+s+s1}{Delta K by K}\PY{l+s+s1}{\PYZsq{}}\PY{p}{,} \PY{l+m+mi}{100} \PY{o}{*} \PY{n}{LC\PYZus{}k\PYZus{}by\PYZus{}k}\PY{p}{)}
  \end{Verbatim}
\end{tcolorbox}

\begin{Verbatim}[commandchars=\\\{\}]
  Total error:  [15.28542981 17.16826619 20.39649475 32.50794362 45.97025081]
  Delta K by K [5.6963217  5.73766894 5.77199133 5.80841662 5.85090121]
\end{Verbatim}

\begin{tcolorbox}[breakable, size=fbox, boxrule=1pt, pad at break*=1mm,colback=cellbackground, colframe=cellborder]
  \prompt{In}{incolor}{6}{\boxspacing}
  \begin{Verbatim}[commandchars=\\\{\}]
    \PY{n}{y} \PY{o}{=} \PY{n}{dH\PYZus{}m\PYZus{}permole}
    \PY{n}{y} \PY{o}{=} \PY{n}{np}\PY{o}{.}\PY{n}{append}\PY{p}{(}\PY{n}{y}\PY{p}{,} \PY{l+m+mi}{0}\PY{p}{)}
    \PY{n}{y} \PY{o}{=} \PY{n}{np}\PY{o}{.}\PY{n}{insert}\PY{p}{(}\PY{n}{y}\PY{p}{,} \PY{l+m+mi}{0}\PY{p}{,} \PY{l+m+mi}{0}\PY{p}{)}

    \PY{n}{x} \PY{o}{=} \PY{n}{x\PYZus{}a}
    \PY{n}{x} \PY{o}{=} \PY{n}{np}\PY{o}{.}\PY{n}{append}\PY{p}{(}\PY{n}{x}\PY{p}{,} \PY{l+m+mi}{1}\PY{p}{)}
    \PY{n}{x} \PY{o}{=} \PY{n}{np}\PY{o}{.}\PY{n}{insert}\PY{p}{(}\PY{n}{x}\PY{p}{,} \PY{l+m+mi}{0}\PY{p}{,} \PY{l+m+mi}{0}\PY{p}{)}

    \PY{n+nb}{print}\PY{p}{(}\PY{n}{x}\PY{p}{,} \PY{n}{y}\PY{p}{)}
  \end{Verbatim}
\end{tcolorbox}

\begin{Verbatim}[commandchars=\\\{\}]
  [0.  0.1 0.3 0.5 0.7 0.9 1. ] [    0.         -1923.98303585 -1534.81998358
  -1088.2122361
  -478.55414415  -301.2563139      0.        ]
\end{Verbatim}

\begin{tcolorbox}[breakable, size=fbox, boxrule=1pt, pad at break*=1mm,colback=cellbackground, colframe=cellborder]
  \prompt{In}{incolor}{7}{\boxspacing}
  \begin{Verbatim}[commandchars=\\\{\}]
    \PY{n}{plt}\PY{o}{.}\PY{n}{plot}\PY{p}{(}\PY{n}{x}\PY{p}{,} \PY{n}{y}\PY{p}{,} \PY{l+s+s1}{\PYZsq{}}\PY{l+s+s1}{o\PYZhy{}k}\PY{l+s+s1}{\PYZsq{}}\PY{p}{)}
    \PY{n}{plt}\PY{o}{.}\PY{n}{title}\PY{p}{(}\PY{l+s+s1}{\PYZsq{}}\PY{l+s+s1}{Enthalpy of mixing vs mole fraction of acetone}\PY{l+s+s1}{\PYZsq{}}\PY{p}{)}
    \PY{n}{plt}\PY{o}{.}\PY{n}{xlabel}\PY{p}{(}\PY{l+s+s1}{\PYZsq{}}\PY{l+s+s1}{Mole fraction of acetone}\PY{l+s+s1}{\PYZsq{}}\PY{p}{)}
    \PY{n}{plt}\PY{o}{.}\PY{n}{ylabel}\PY{p}{(}\PY{l+s+s1}{\PYZsq{}}\PY{l+s+s1}{Enthalpy of mixing (J/mol)}\PY{l+s+s1}{\PYZsq{}}\PY{p}{)}
  \end{Verbatim}
\end{tcolorbox}

\begin{tcolorbox}[breakable, size=fbox, boxrule=.5pt, pad at break*=1mm, opacityfill=0]
  \prompt{Out}{outcolor}{7}{\boxspacing}
  \begin{Verbatim}[commandchars=\\\{\}]
    Text(0, 0.5, 'Enthalpy of mixing (J/mol)')
  \end{Verbatim}
\end{tcolorbox}

\begin{center}
  \adjustimage{max size={0.7\linewidth}{0.7\paperheight}}{output_7_1.png}
\end{center}
{ \hspace*{\fill} \\}

\begin{tcolorbox}[breakable, size=fbox, boxrule=1pt, pad at break*=1mm,colback=cellbackground, colframe=cellborder]
  \prompt{In}{incolor}{8}{\boxspacing}
  \begin{Verbatim}[commandchars=\\\{\}]
    \PY{n}{a\PYZus{}BSpline} \PY{o}{=} \PY{n}{interpolate}\PY{o}{.}\PY{n}{make\PYZus{}interp\PYZus{}spline}\PY{p}{(}\PY{n}{x}\PY{p}{,} \PY{n}{y}\PY{p}{)}

    \PY{n}{x\PYZus{}inter} \PY{o}{=} \PY{n}{np}\PY{o}{.}\PY{n}{linspace}\PY{p}{(}\PY{l+m+mi}{0}\PY{p}{,} \PY{l+m+mi}{1}\PY{p}{,} \PY{l+m+mi}{300}\PY{p}{)}
    \PY{n}{y\PYZus{}inter} \PY{o}{=} \PY{n}{a\PYZus{}BSpline}\PY{p}{(}\PY{n}{x\PYZus{}inter}\PY{p}{)}

    \PY{n}{plt}\PY{o}{.}\PY{n}{plot}\PY{p}{(}\PY{n}{x\PYZus{}inter}\PY{p}{,} \PY{n}{y\PYZus{}inter}\PY{p}{,} \PY{l+s+s1}{\PYZsq{}}\PY{l+s+s1}{\PYZhy{}b}\PY{l+s+s1}{\PYZsq{}}\PY{p}{)}
    \PY{n}{plt}\PY{o}{.}\PY{n}{title}\PY{p}{(}\PY{l+s+s1}{\PYZsq{}}\PY{l+s+s1}{Enthalpy of mixing vs mole fraction of acetone (after interpolation)}\PY{l+s+s1}{\PYZsq{}}\PY{p}{)}
    \PY{n}{plt}\PY{o}{.}\PY{n}{xlabel}\PY{p}{(}\PY{l+s+s1}{\PYZsq{}}\PY{l+s+s1}{Mole fraction of acetone}\PY{l+s+s1}{\PYZsq{}}\PY{p}{)}
    \PY{n}{plt}\PY{o}{.}\PY{n}{ylabel}\PY{p}{(}\PY{l+s+s1}{\PYZsq{}}\PY{l+s+s1}{Enthalpy of mixing (J/mol)}\PY{l+s+s1}{\PYZsq{}}\PY{p}{)}
  \end{Verbatim}
\end{tcolorbox}

\begin{tcolorbox}[breakable, size=fbox, boxrule=.5pt, pad at break*=1mm, opacityfill=0]
  \prompt{Out}{outcolor}{8}{\boxspacing}
  \begin{Verbatim}[commandchars=\\\{\}]
    Text(0, 0.5, 'Enthalpy of mixing (J/mol)')
  \end{Verbatim}
\end{tcolorbox}

\begin{center}
  \adjustimage{max size={0.7\linewidth}{0.7\paperheight}}{output_8_1.png}
\end{center}
{ \hspace*{\fill} \\}

\begin{tcolorbox}[breakable, size=fbox, boxrule=1pt, pad at break*=1mm,colback=cellbackground, colframe=cellborder]
  \prompt{In}{incolor}{9}{\boxspacing}
  \begin{Verbatim}[commandchars=\\\{\}]
    \PY{k}{def} \PY{n+nf}{calcSlope}\PY{p}{(}\PY{n}{x}\PY{p}{,} \PY{n}{y}\PY{p}{,} \PY{n}{p}\PY{p}{)}\PY{p}{:}
    \PY{k}{return} \PY{p}{(}\PY{n}{y}\PY{p}{[}\PY{n}{p}\PY{o}{+}\PY{l+m+mi}{1}\PY{p}{]} \PY{o}{\PYZhy{}} \PY{n}{y}\PY{p}{[}\PY{n}{p}\PY{p}{]}\PY{p}{)} \PY{o}{/} \PY{p}{(}\PY{n}{x}\PY{p}{[}\PY{n}{p}\PY{o}{+}\PY{l+m+mi}{1}\PY{p}{]} \PY{o}{\PYZhy{}} \PY{n}{x}\PY{p}{[}\PY{n}{p}\PY{p}{]}\PY{p}{)}

    \PY{n+nb}{print}\PY{p}{(}\PY{l+s+s1}{\PYZsq{}}\PY{l+s+s1}{x: 0, slope: }\PY{l+s+s1}{\PYZsq{}}\PY{p}{,} \PY{n}{calcSlope}\PY{p}{(}\PY{n}{x\PYZus{}inter}\PY{p}{,} \PY{n}{y\PYZus{}inter}\PY{p}{,} \PY{l+m+mi}{0}\PY{p}{)}\PY{p}{)}
    \PY{n+nb}{print}\PY{p}{(}\PY{l+s+s1}{\PYZsq{}}\PY{l+s+s1}{x: 0.5, slope: }\PY{l+s+s1}{\PYZsq{}}\PY{p}{,} \PY{n}{calcSlope}\PY{p}{(}\PY{n}{x\PYZus{}inter}\PY{p}{,} \PY{n}{y\PYZus{}inter}\PY{p}{,} \PY{l+m+mi}{149}\PY{p}{)}\PY{p}{)}
    \PY{n+nb}{print}\PY{p}{(}\PY{l+s+s1}{\PYZsq{}}\PY{l+s+s1}{x: 1, slope: }\PY{l+s+s1}{\PYZsq{}}\PY{p}{,} \PY{n}{calcSlope}\PY{p}{(}\PY{n}{x\PYZus{}inter}\PY{p}{,} \PY{n}{y\PYZus{}inter}\PY{p}{,} \PY{l+m+mi}{298}\PY{p}{)}\PY{p}{)}
  \end{Verbatim}
\end{tcolorbox}

\begin{Verbatim}[commandchars=\\\{\}]
  x: 0, slope:  -31521.026814679495
  x: 0.5, slope:  2294.1720273082487
  x: 1, slope:  4913.551030789136
\end{Verbatim}

% Add a bibliography block to the postdoc



\end{document}
